\documentclass[slidestop,compress,blackandwhite]{beamer}

\usepackage[utf8x]{inputenc}
\usepackage[italian]{babel}
\usepackage{tikz}
\usepackage{nameref}
\usepackage{scrextend}

\usetheme{Antibes}
\usecolortheme{default}

\setbeameroption{show notes}


\pgfdeclareimage[height=1.5cm]{logo}{imgs/unipd_logo}

\setbeamercolor{block title}{fg=red,bg=structure!15}

\setbeamercolor{block body}{bg=structure!15}

% TITOLO
\setbeamercovered{transparent}

\author{Moreno Ambrosin}

\title[Progetto esame di Sistemi Concorrenti e Distribuiti]{Railway Simulation}

\institute[\insertframenumber/\inserttotalframenumber]{
	\large{Università degli studi di Padova} \\
	\vspace{5pt}
	\normalsize Facoltà di Scienze MM. FF. NN. \\
	\vspace{5pt}
	\small Corso di laurea in Informatica
}
\date{Settembre 2013}

\newcommand{\itemB}[3]{
	\item \textbf{#1} #2 \vspace{#3}
}

\newcommand{\ttt}[1]{\texttt{#1}}
\newcommand{\ii}[1]{\textit{#1}}
\newcommand{\bb}[1]{\textbf{#1}}


\newcommand{\treno}{\ii{treno}}
\newcommand{\treni}{\ii{treni}}
\newcommand{\viaggiatore}{\ii{viaggiatore}}
\newcommand{\viaggiatori}{\ii{viaggiatori}}
\newcommand{\stazione}{\ii{stazione}}
\newcommand{\stazioni}{\ii{stazioni}}
\newcommand{\piattaforma}{\ii{piattaforma}}
\newcommand{\piattaforme}{\ii{piattaforme}}
\newcommand{\ticket}{\ii{biglietto}}
\newcommand{\tickets}{\ii{biglietti}}
\newcommand{\segmento}{\ii{segmento}}
\newcommand{\segmenti}{\ii{segmenti}}
\newcommand{\route}{\ii{percorso}}
\newcommand{\routes}{\ii{percorsi}}
\newcommand{\stage}{\ii{tappa}}
\newcommand{\stages}{\ii{tappe}}
\newcommand{\biglietteria}{\ii{biglietteria}}
\newcommand{\biglietterie}{\ii{biglietterie}}
\newcommand{\timetable}{\ii{tabella oraria}}
\newcommand{\timetables}{\ii{tabelle di orari}}
\newcommand{\controller}{\ii{controllo centrale}}
\newcommand{\regione}{\ii{regione}}
\newcommand{\regioni}{\ii{regioni}}
\newcommand{\gateway}{\ii{stazione di gateway}}
\newcommand{\gateways}{\ii{stazioni di gateway}}

\newcommand{\PRO}{\textbf{PRO:}}
\newcommand{\CONTRO}{\textbf{CONTRO:}}

%\renewcommand{\item}{\vspace{0.1cm}\item}

\newcommand{\cm}[1]{\vspace{#1cm}}

\newcommand{\describe}[2]{
	\textbf{#1}\\
	\begin{addmargin}[2em]{0em}
		#2
	\end{addmargin}
}


\newcommand{\newtitle}[4]{
	#1 
	\ifx&#2&%
	\else
  		\large- #2
	\fi
	\ifx&#3&%
	\else
  		\normalsize- #3
	\fi
	\ifx&#4&%
	\else
  		\normalsize (#4)
	\fi
}

\newcommand{\newframe}[5]{
	\begin{frame}
		\frametitle{\newtitle{#1}{#2}{#3}{#4}}
		#5
	\end{frame}
}

\newcommand{\itemt}[1]{\item (\ttt{#1})}

\newcommand{\myitemize}[1]{
	\begin{itemize}\itemsep4pt
	#1
	\end{itemize}
}

\newcommand{\newfigure}[3]{
	\begin{figure}
		\centering
		\includegraphics[scale=#2]{#1}
		\ifx&#3&%
		\else
	  		\caption{\scriptsize #3}
		\fi
	\end{figure}
}


\begin{document}
	
	\usebackgroundtemplate{
		\hspace{0.13\paperwidth}\includegraphics[height=\paperheight]{imgs/logoUnipd}
	}	
	
	\begin{frame}[c]
		\titlepage
	\end{frame}
	
	\newframe{Indice}{}{}{}{
		\tableofcontents
	}
	
	\setbeamertemplate{footline}[text line]{\parbox{\linewidth}{\vspace*{-8pt}\scriptsize\insertframenumber/\inserttotalframenumber\hfill Progetto Sistemi Concorrenti e Distribuiti\hfill}}
	
	
	
% ###################################################################################################################
\section{Introduzione}\label{intro}

	\newframe{\nameref{intro}}{}{}{}{
		\myitemize {
			\item La progettazione di un sistema distribuito si compone di:
				\cm{0.5}
				\myitemize {
					\item Analisi del problema.
						\cm{0.4}
						\myitemize {
							\item Aspetti legati alla Distribuzione.
							\item Aspetti legati alla Concorrenza.
						}
					\cm{0.4}
					\item Costruzione di una soluzione.
					\cm{0.4}
					\item Scelta delle Tecnologie.
					\cm{0.4}
				}
		}
	}

\section{Analisi del Problema}\label{analisis}
	\newframe{\nameref{analisis}}{}{}{}{
		\myitemize {
			\item Identificare e definire i requisiti di massima del sistema.
				\myitemize{
					\item Operazione sottovalutata ma importante.
				}
%			\item I requisiti possono variare durante la progettazione (grado di libertà eleveto nella specifica).
%				\myitemize {
%					\item Importante mantenere semplice la specifica (si può poi procedere in maniera incrementale).
%				}
			\item Individuazione e prima definizione delle entità del sistema.
				\myitemize {
					\item ad es. nel progetto di un sistema ferroviario:
						\myitemize{
							\item Treno (entità attiva)
							\item Viaggiatore (antità attiva)
							\item Segmento (entità reattiva)
							\item Piattaforma (entità reattiva)
							\item Biglietteria (entità reattiva)
							\item Stazione (entità reattiva)
						}
				}
		}
	}
	
	\newframe{\nameref{analisis}}{Distribuzione}{}{1}{
		\myitemize {
			\item Primo aspetto da valutare
				\myitemize {
					\item Fornisce un'architettura di alto livello del sistema.
					\item Alcune scelte vincolano le modalità di interazione tra le entità.
				}
			\item Caratteristiche Desiderabili
				\myitemize {
					\item Il sistema dovrà apparire agli utenti in modo unitario e coerente. 
						\myitemize{
							\item ad es. nel progetto di un sistema ferroviario, la componente di visualizzazione dovrà nascondere l'architettura di distribuzione. 
						}
					\itemB{Trasparenza:}{Il sistema dovra il più possibile rendere trasparenti all'utente le caratteristiche legate alla distribuzione (Accesso, Collocazione, Migrazione, Spostamento, Replicazione, Malfunzionamento, Persistenza}{0.1cm}
				}
		}
	}
	
	\newframe{\nameref{analisis}}{Distribuzione}{}{2}{
		\myitemize {
			\item[]
				\myitemize {
					\itemB{Openess:}{
						\myitemize {
							\item Il sistema dovrà garantire portabilità e interoperabilità.
							\item Il sistema dovrà essere fruibile mediante regole standard (interfacce).
							\item Organizzazione del sistema in componenti di dimensione ridotta, e facilmente sostituibili.
							\item Separazione tra \ii{politiche} e \ii{meccanismi}.
						}
					}{0.1cm}
					\itemB{Scalabilità:}{
						\myitemize {
							\item Rispetto alla cardinalità del sistema (ad es. nel progetto di un sistema ferroviario, è desiderabile poter aumentare la popolazione di Stazioni e Segmenti di collegamento).
							\item Rispetto alla distribuzione spaziale delle componenti.
							\item Rispetto alle problematiche locali di gestione (che non devono affliggere l'intero sistema).
						}
					}{0.1cm}
				}
		}
	}
	
	\newframe{\nameref{analisis}}{Distribuzione}{}{3}{
		\myitemize {
			\item[]
				\myitemize {
					\itemB{Fault Tollerance:}{
						\myitemize{
							\item Il sistema deve essere progettato in modo tale da ridurre l'impatto causato da \ii{partial failures}.
							\item Il sistema dovrà gestire errori di comunicazione tra i nodi.
						}
					}{0.1cm}
					
					\itemB{Avvio ordinato:}{Il sistema dovrà essere avviato in modo tale da permettere a tutte le componenti di comunicare senza errori.}{0.1cm}
					\itemB{Terminazione in stato Consistente}{Il sistema deve poter essere terminato in uno stato consistente; nessun entità dovrà rimanere attiva dopo la procedura di terminazione.}{0.1cm}
				}
				
		}
	}
	
	\newframe{\nameref{analisis}}{Distribuzione}{Scelte di progetto}{}{
		\myitemize {
			\item Modellazione ad alto livello delle componenti del sistema necessarie e loro distribuzione.
			\item Scelta delle modalità di distribuzione
				\myitemize{
					\item Dove adottare distribuzione \ii{verticale} o \ii{orizzontale}
				}
			\item Scelta della modalità di comunicazione tra le componenti (\ii{sincrona} o \ii{asincrona}) e definizione di possibili interfacce.
			\item Valutazione delle implicazioni nell'adozione di gradi di distribuzione diversi sul sistema.
			\item Individuazione delle problematiche specifiche del problema, ad es.
				\myitemize {
					\item Gestione del \ii{Name Resolution} per le componenti.
					\item Gestione della sincronizzazione tra clock fisici dei nodi che ospitano le varie componenti.
				}
			
		}
	}
	
	\newframe{\nameref{analisis}}{Distribuzione}{Esempio}{1}{
		\newfigure{imgs/All_distributed}{0.22}{Architettura di distribuzione in cui tutte le entità principali vengono distribuite.}
		\note[includegraphics]{
			PRO:
			\myitemize {
				\item Scalabilità 
				\item Fault Tollerance
			}
		}
		\note[includegraphics]{CONTRO
			\myitemize {
				\item Complessità terminazione
				\item Assenza di riferimento temporale
			}
		}
		\note[includegraphics]{Deve essere adottato un sistema di Naming che scali meglio di semplice tabella [Nome,Indirizzo]}
		
	}
	
	\newframe{\nameref{analisis}}{Distribuzione}{Esempio}{2}{
		\cm{0.5}
		\newfigure{imgs/nothing_distributed}{0.22}{Architettura di distribuzione in cui solo Controller Centrale e Visualizzazione sono distribuite.}
	}
	
	\newframe{\nameref{analisis}}{Distribuzione}{Esempio}{3}{
		\newfigure{imgs/solution}{0.24}{Architettura di distribuzione in cui sono aggiunte e distribuite \ii{regioni}.}
	}

%############################### CONCURRENCY	

	
	\newframe{\nameref{analisis}}{Concorrenza}{}{}{
		
		\myitemize {
			\item Scelta delle caratteristiche specifiche per ciascuna entità.
				\myitemize {
					\item ad es.: scelta di definire un Segmento come entità reattiva con agente di controllo, ad accesso mutuamente esclusivo con molteplicità $N\ge1$.
					\item ad es.: scelta di definire una Stazione come una struttura che mantiene al suo interno
						\myitemize {
							\item un numero $M\ge1$ di Piattaforme (entità reattive con agente di controllo ad accesso mutuamente esclusivo con molteplicità $1$),	....
						}
				}
			\item Prima definizione dei protocolli logici di interazione tra entità che risiedono sullo stesso nodo di calcolo.
				\myitemize {
					\item Il più possibile indipendente dalla scelta di un modello di concorrenza specifico.
					\item Identificazione dei punti critici in cui il problema è concorrente.
				}
				
		}
		
	}
	
	\newframe{\nameref{analisis}}{Concorrenza}{Esempio}{1}{
		\newfigure{imgs/ingresso_segmento}{0.5}{Ingresso in un segmento}
	}
	
	\newframe{\nameref{analisis}}{Concorrenza}{Esempio}{2}{
		\newfigure{imgs/ingresso_stazione}{0.5}{Ingresso in un segmento}
	}
	
	
% ####################################### PROBLEM SOLUTION ###################################
	
\section{Soluzione al Problema}\label{sol}

	\newframe{\nameref{sol}}{Distribuzione}{}{} {
		\myitemize {
			\item Scelta di un'architettura di distribuzione.
				\myitemize {
					\item Introduce nuove entità?
					\item ad es.: utilizzare Regioni come collezione di Stazioni e Segmenti, che risiedono su un singolo nodo di calcolo, e sulle quali transitano Treni e Viaggiatori.
					\item ad es.: introduzione di una gerarchia di Biglietterie per distribuire conoscenza e oneri di calcolo.
						\myitemize {
							\item Centrale
							\item Regionale
							\item di Stazione
						}
				}
			\item Identificazione delle problematiche conseguenti a tale scelta.
				
		}
	}

	\newframe{\nameref{sol}}{Distribuzione}{Esempio}{} {
		\myitemize {
			\item Come realizzare il passaggio di entità Treno e Viaggiatore tra Regioni?
			\myitemize {
				\item \ii{possibile soluzione:} Utilizzare Stazioni speciali (di ``gateway'') per permettere l'uscita di un Treno da una Regione; trasferimento diretto di un Viaggiatore.
			}
			\item Come si traduce il trasferimento remoto di una entità?
				\myitemize {
					\item creazione/distruzione?
					\item replicazione?
					\item Vincolo sulla realizzazione dell'entità per facilitare il trasferimento remoto!
				}
		}
	}
	
	\newframe{\nameref{sol}}{Distribuzione}{Avvio e Terminazione}{1}{
		\describe{Avvio}{
			\myitemize{
				\item Deve essere coordinato tra le entità.
				\item \`E opportuno scegliere un ordine di avvio tra le componenti e separare la fase di inizializzazione della componente e l'avvio della simulazione.
			}
		}
		\cm{0.1}
		\describe{Terminazione}{
			\myitemize{
				\item Ha come pre-requisito la definizione dei limiti entro i quali uno Stato del sistema è consistente. Ad es. nel progetto di un sistema ferroviario:
					\myitemize {
						\item è accettabile che il sistema termini con un certo numero di Treni in attesa di accedere ad una Piattaforma;
						\item non è accettato lo stato di terminazione per il quale un Viaggiatore è in attesa di un Biglietto.
					} 
			}
		}
	}
	
	\newframe{\nameref{sol}}{Distribuzione}{Avvio e Terminazione}{2}{
		\describe{}{
			\myitemize {
				\item \`E conveniente adattare algoritmi noti distribuiti (ad es. \ii{distributed snapshot}).
				\item Nessun thread in esecuzione sui nodi di calcolo dopo la procedura.
			}
		}
	}
	\newframe{\nameref{sol}}{Concorrenza}{}{}{
		
		\myitemize {
			\item Scelta di un modello di concorrenza adatto alle caratteristiche del problema.
			\item Valutazione di modelli differenti
				\myitemize{
					\item ad es. modello ad \ii{Attori} o a \ii{monitor}.
				}
			\item Modellazione delle entità di simulazione e della loro interazione con strumenti di modello.
				\myitemize {
					\item Scomposizione delle interazioni in sottoproblemi semplici.
				}
			\item Evitare scelte di progettazione che utilizzano operazioni specifiche offerte dalle tecnologie
				\myitemize {
					\item Nessuna assunzione a priori sul linguaggio che verrà utilizzato.
				}
			\item Validare ciascuna soluzione, partendo dai pre-requisiti.
				
		}
		
	}
	
	\newframe{\nameref{sol}}{Tecnologiche}{}{}{
		\myitemize {
			\item Scelta delle tecnologie che meglio si adattano alle scelte di progetto.
			\item Interessante l'utilizzo di tecnologie eterogenee
				\myitemize {
					\item \`E difficile pensare ad un Sistema Distribuito realizzato con tecnologie omogenee.
					\item Possibilità di utilizzare tecnologie specifiche per singola componente
						\myitemize {
							\item Nella solouzione che ho adottato, ho utilizzato il linguaggio Ada per codificare le componenti che rappresentano le Regioni;
							\item Utilizzo del linguaggio Scala per la realizzazione di Name Server, Biglietteria e Controller Centrale;
							\item Utilizzo di Javascript e HTML per la realizzazione dell'interfaccia grafica.
						}
				}
		}
	}
\section{Conclusioni}\label{conclusions}
	
	\newframe{\nameref{conclusions}}{}{}{}{
		
	}
	
\end{document}

