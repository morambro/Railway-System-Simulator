\documentclass[slidestop,compress,blackandwhite]{beamer}

\usepackage[utf8x]{inputenc}
\usepackage[italian]{babel}
\usepackage{tikz}
\usepackage{nameref}
\usepackage{scrextend}
\usepackage{listings}
\usepackage{ragged2e}

\usetheme{Antibes}
\usecolortheme{default}

%\setbeameroption{show notes}


\pgfdeclareimage[height=1.5cm]{logo}{imgs/unipd_logo}

\setbeamercolor{block title}{fg=red,bg=structure!15}

\setbeamercolor{block body}{bg=structure!15}

% TITOLO
\setbeamercovered{transparent}

\author{Moreno Ambrosin}

\title[Progetto di Sistemi Concorrenti e Distribuiti]{Realizzazione del Progetto di Sistemi Concorrenti e Distribuiti}

\institute[\insertframenumber/\inserttotalframenumber]{
	\large{Università degli studi di Padova} \\
	\vspace{5pt}
	\normalsize Facoltà di Scienze MM. FF. NN. \\
	\vspace{5pt}
	\small Corso di laurea in Informatica
}
\date{Agosto 2013}

\newcommand{\itemB}[3]{
	\item \textbf{#1} #2 \vspace{#3}
}

\newcommand{\ttt}[1]{\texttt{#1}}
\newcommand{\ii}[1]{\textit{#1}}
\newcommand{\bb}[1]{\textbf{#1}}


\newcommand{\treno}{\ii{treno}}
\newcommand{\treni}{\ii{treni}}
\newcommand{\viaggiatore}{\ii{viaggiatore}}
\newcommand{\viaggiatori}{\ii{viaggiatori}}
\newcommand{\stazione}{\ii{stazione}}
\newcommand{\stazioni}{\ii{stazioni}}
\newcommand{\piattaforma}{\ii{piattaforma}}
\newcommand{\piattaforme}{\ii{piattaforme}}
\newcommand{\ticket}{\ii{biglietto}}
\newcommand{\tickets}{\ii{biglietti}}
\newcommand{\segmento}{\ii{segmento}}
\newcommand{\segmenti}{\ii{segmenti}}
\newcommand{\route}{\ii{percorso}}
\newcommand{\routes}{\ii{percorsi}}
\newcommand{\stage}{\ii{tappa}}
\newcommand{\stages}{\ii{tappe}}
\newcommand{\biglietteria}{\ii{biglietteria}}
\newcommand{\biglietterie}{\ii{biglietterie}}
\newcommand{\timetable}{\ii{tabella oraria}}
\newcommand{\timetables}{\ii{tabelle di orari}}
\newcommand{\controller}{\ii{controllo centrale}}
\newcommand{\regione}{\ii{regione}}
\newcommand{\regioni}{\ii{regioni}}
\newcommand{\gateway}{\ii{stazione di gateway}}
\newcommand{\gateways}{\ii{stazioni di gateway}}

\newcommand{\PRO}{\textbf{PRO:}}
\newcommand{\CONTRO}{\textbf{CONTRO:}}

%\renewcommand{\item}{\vspace{0.1cm}\item}

\newcommand{\cm}[1]{\vspace{#1cm}}

\newcommand{\describe}[2]{
	\textbf{#1}\\
	\begin{addmargin}[2em]{0em}
		#2
	\end{addmargin}
}


\newcommand{\newtitle}[4]{
	#1 
	\ifx&#2&%
	\else
  		\large- #2
	\fi
	\ifx&#3&%
	\else
  		\normalsize- #3
	\fi
	\ifx&#4&%
	\else
  		\normalsize (#4)
	\fi
}

\newcommand{\newframe}[5]{
	\begin{frame}
		\frametitle{\newtitle{#1}{#2}{#3}{#4}}
		#5
	\end{frame}
}

\newcommand{\itemt}[1]{\item (\ttt{#1})}

\newcommand{\myitemize}[1]{
	\begin{itemize}\itemsep4pt
	#1
	\end{itemize}
}

\newcommand{\newfigure}[3]{
	\begin{figure}
		\centering
		\includegraphics[scale=#2]{#1}
		\ifx&#3&%
		\else
	  		\caption{\scriptsize #3}
		\fi
	\end{figure}
}


\begin{document}
	
	\usebackgroundtemplate{
		\hspace{0.13\paperwidth}\includegraphics[height=\paperheight]{imgs/logoUnipd}
	}	
	
	\begin{frame}[c]
		\titlepage
	\end{frame}
	
	\newframe{Indice}{}{}{}{
		\tableofcontents
	}
	
	\setbeamertemplate{footline}[text line]{\parbox{\linewidth}{\vspace*{-8pt}\scriptsize\insertframenumber/\inserttotalframenumber\hfill Progetto di Sistemi Concorrenti e Distribuiti\hfill}}
	
	
	
% ###################################################################################################################
\section{Introduzione}\label{intro}

	\newframe{\nameref{intro}}{}{}{}{
		\myitemize {
			\item La progettazione di un sistema distribuito si compone di:
				\cm{0.3}
				\myitemize {
					\item Analisi del problema.
						\cm{0.3}
						\myitemize {
							\item Definizione della specifica.
							\item Analisi degli aspetti legati alla Distribuzione.
							\item Analisi degli aspetti legati alla Concorrenza.
						}
					\cm{0.3}
					\item Costruzione di una soluzione.
					\cm{0.3}	
						\myitemize {
							\item Definizione di una architettura di distribuzione.
							\item Risoluzione delle problematiche di concorrenza.
						}
					\cm{0.3}
					\item Scelta delle Tecnologie.
				}
		}
	}

\section{Analisi del Problema}\label{analisis}
	\newframe{\nameref{analisis}}{}{}{}{
		\myitemize {
%			\item I requisiti possono variare durante la progettazione (grado di libertà eleveto nella specifica).
%				\myitemize {
%					\item Importante mantenere semplice la specifica (si può poi procedere in maniera incrementale).
%				}
			\item Individuazione e prima definizione delle entità del Sistema.
				\myitemize {
					\item ad es. nel progetto di un sistema ferroviario:
						\myitemize{
							\item Treno (entità attiva)
							\item Viaggiatore (entità attiva)
							\item Segmento (entità reattiva)
							\item Stazione (entità passiva)\\\cm{0.1}
								- Piattaforma (entità reattiva)\\ \cm{0.1}
								- Biglietteria (entità reattiva)\\ \cm{0.1}
								- Pannello Informativo (entità reattiva)
							
						}
				}
			\item Identificazione e definizione dei requisiti di massima del Sistema.
				\myitemize{
					\item Operazione sottovalutata ma importante.
				}
		}
	}
	
	\newframe{\nameref{analisis}}{Distribuzione}{}{1}{
		\myitemize {
			\item Primo aspetto da valutare
				\myitemize {
					\item Decomposizione architetturale del Sistema (derivazione di un'architettura di Sistema, di alto livello).
					\item Alcune scelte vincolano le modalità di interazione tra le entità.
				}
			
			\item Il sistema dovrà apparire agli utenti come unitario e coerente. 
				
			\item Caratteristiche Desiderabili
				\myitemize {
					\itemB{Trasparenza:}{Il sistema dovrà il più possibile rendere trasparenti all'utente le caratteristiche legate alla distribuzione (Accesso, Collocazione, Migrazione, Spostamento, Replicazione, Malfunzionamento, Persistenza)}{0.1cm}
				}
		}
	}
	
	\newframe{\nameref{analisis}}{Distribuzione}{}{2}{
		\myitemize {
			\item[]
				\myitemize {
					\itemB{Openess:}{\justifying
						\myitemize {
							\item Il sistema dovrà garantire portabilità e interoperabilità.
							\item Il sistema dovrà essere fruibile mediante regole standard (interfacce).
							\item \justifying Organizzazione del sistema in componenti di dimensione ridotta, e facilmente sostituibili.
							\item Separazione tra \ii{politiche} e \ii{meccanismi}.
						}
					}{0.1cm}
					\itemB{Scalabilità:}{\justifying
						\myitemize {
							\item Rispetto alla cardinalità del sistema (ad es. nel progetto di un sistema ferroviario, è desiderabile poter aumentare la popolazione di Stazioni e Segmenti di collegamento).
							\item Rispetto alla distribuzione spaziale delle componenti.
							\item Rispetto alle problematiche locali di gestione (che non devono affliggere l'intero sistema).
						}
					}{0.1cm}
				}
		}
	}
	
	\newframe{\nameref{analisis}}{Distribuzione}{}{3}{
		\myitemize {
			\item[]
				\myitemize {
					\itemB{Fault Tollerance:}{
						\myitemize{
							\item Il sistema deve essere progettato in modo tale da ridurre l'impatto causato da \ii{partial failures}.
							\item Il sistema dovrà gestire errori di comunicazione tra i nodi.
						}
					}{0.1cm}
					
					\itemB{Avvio ordinato:}{\justifying Il sistema dovrà essere avviato in modo tale da permettere a tutte le componenti di comunicare senza errori.}{0.1cm}
					\itemB{Terminazione in stato Consistente}{\justifying Il sistema deve poter essere terminato in uno stato consistente; nessun entità dovrà rimanere attiva dopo la procedura di terminazione.}{0.1cm}
				}
				
		}
	}
	
	\newframe{\nameref{analisis}}{Distribuzione}{}{4}{
		\myitemize {
			\item Prima modellazione ad alto livello delle componenti distribuite del Sistema.
			\item Scelta di alcuni aspetti legati alla distribuzione
				\myitemize{
					\item dove adottare distribuzione \ii{verticale} o \ii{orizzontale};
					\item modalità di comunicazione tra le componenti (\ii{sincrona} o \ii{asincrona});
					\item definizione di possibili interfacce.		
				}
			\item Valutazione delle implicazioni nell'adozione di gradi di distribuzione diversi sul sistema.
			\item Individuazione delle problematiche specifiche del problema.%, ad es.
%				\myitemize {
%					\item Gestione del \ii{Name Resolution} per le componenti.
%					\item Gestione della sincronizzazione tra clock fisici dei nodi che ospitano le varie componenti.
%				}
			
		}
	}
	
	\newframe{\nameref{analisis}}{Distribuzione}{Esempio}{1}{
		\newfigure{imgs/All_distributed}{0.22}{Architettura di alto livello in cui tutte le entità principali vengono distribuite.}		
	}
	
	\newframe{\nameref{analisis}}{Distribuzione}{Esempio}{2}{
		\newfigure{imgs/nothing_distributed}{0.25}{Architettura di alto livello in cui solo Controller Centrale e Visualizzazione sono distribuite.}
	}
	
	\newframe{\nameref{analisis}}{Distribuzione}{Esempio}{3}{
		\newfigure{imgs/solution}{0.24}{Architettura di alto livello con distribuzione a livello di \ii{Regioni}.}
	}

%############################### CONCURRENCY	

	
	\newframe{\nameref{analisis}}{Concorrenza}{}{}{
		
		\myitemize {
			\item Prima definizione dei protocolli logici di interazione concorrente tra le entità.
				\myitemize {
					\item Il più possibile indipendente dalla scelta di un modello di concorrenza specifico.
					\item Identificazione dei punti critici in cui il problema è concorrente.
				}
			\item Definizione delle caratteristiche specifiche per ciascuna entità.
				\myitemize {
					\item ad es.: Segmento come entità reattiva con agente di controllo, a molteplicità $N\ge1$.
					\item ad es.: Stazione come una entità passiva, che mantiene al suo interno
						\myitemize {
							\item un numero $M\ge1$ di Piattaforme (entità reattive con agente di controllo, a molteplicità $1$),	....
						}
				}
				
		}
		
	}
	
	\newframe{\nameref{analisis}}{Concorrenza}{Esempio}{1}{
		\newfigure{imgs/ingresso_segmento}{0.5}{Accesso ad un segmento.}
	}
	
	\newframe{\nameref{analisis}}{Concorrenza}{Esempio}{2}{
		\newfigure{imgs/ingresso_stazione}{0.5}{Uscita da un segmento e accesso alla Piattaforma successiva.}
	}
	
	
% ####################################### PROBLEM SOLUTION ###################################
	
\section{Costruzione di una Soluzione}\label{sol}

	\newframe{\nameref{sol}}{Distribuzione}{}{} {
		\myitemize {
			\item Scelta di un'architettura di distribuzione.
			\item Introduzione di nuove entità
				\myitemize {
					\item ad es.: utilizzo di Regioni.
				}
			\item Decomposizione architetturale a grana più fine
				\myitemize {
					\item ad es.: introduzione di una gerarchia di Biglietterie per distribuire conoscenza e oneri di calcolo.
						\myitemize {
							\item Centrale
							\item Regionale
							\item Interna alle Stazioni
						}
				}
			\item Scelta del modello di comunicazione tra le componenti.
			\item Identificazione delle problematiche conseguenti alle scelte architetturali.
				
		}
	}

	\newframe{\nameref{sol}}{Distribuzione}{Esempio}{} {
		\myitemize {
			\item Come realizzare il passaggio di entità Treno e Viaggiatore tra Regioni?
				\myitemize {
					\item \ii{possibile soluzione:} Utilizzare Stazioni speciali (di ``gateway'') per permettere l'uscita di un Treno da una Regione; trasferimento diretto di un Viaggiatore.
				}
			\item Come si traduce il trasferimento remoto di una entità?
				\myitemize {
					\item creazione/distruzione?
					\item replicazione?
					\item Vincolo sulla realizzazione dell'entità per facilitare il trasferimento remoto!
					\item \ii{possibile soluzione:} Disaccoppiamento tra entità e thread che ne esegue le operazioni: utilizzo di un thread pool e di descrittori di entità.
				}
		}
	}
	
	\newframe{\nameref{sol}}{Distribuzione}{Avvio e Terminazione}{1}{
		\justifying
		\describe{Avvio}{
			\myitemize{
				\item Deve essere coordinato tra le entità.
					\myitemize {
						\item Devono essere evitati tentativi di comunicazione tra componenti non ancora pronte o allocate.
					}
				\item \`E opportuno scegliere un ordine di avvio tra le componenti e separare la fase di inizializzazione della componente e l'avvio della simulazione.
			}
		}
		\cm{0.1}
		\describe{Terminazione}{
			\myitemize{
				\item Ha come prerequisito la definizione dei limiti entro i quali uno Stato del sistema è consistente. Ad es. nel progetto di un sistema ferroviario:
					\myitemize {
						\item è accettabile che il sistema termini con un certo numero di Treni in attesa di accedere ad una Piattaforma;
					} 
			}
		}
	}
	
	\newframe{\nameref{sol}}{Distribuzione}{Avvio e Terminazione}{2}{
		\describe{}{
			\myitemize {
				\item[]
					\myitemize {
						\item non è accettato lo stato di terminazione per il quale un Viaggiatore è in attesa di un Biglietto.
					}
				\item \`E conveniente adattare algoritmi distribuiti noti (ad es. \ii{distributed snapshot}).
				\item Nessun thread in esecuzione sui nodi di calcolo dopo la procedura.
			}
		}
	}
	
	
	\newframe{\nameref{sol}}{Concorrenza}{}{}{
		\justifying
		\myitemize {
			\item Scelta di un modello di concorrenza adatto alle caratteristiche del problema.
			\item Valutazione di modelli differenti
				\myitemize{
					\item ad es. modello ad \ii{Attori} o a \ii{monitor}.
				}
			\item Modellazione delle entità di simulazione e della loro interazione con strumenti di modello.
				\myitemize {
					\item Scomposizione delle interazioni in sottoproblemi semplici.
				}
			\item Evitare scelte di progettazione che utilizzano operazioni specifiche offerte dalle tecnologie
				\myitemize {
					\item Nessuna assunzione a priori sul linguaggio che verrà utilizzato.
					\item Nessuna assunzione sulle politiche di scheduling adottate dalla macchina sottostante.
				}
			\item Attenzione a \ii{deadlock} e \ii{starvation}.
%			\item Validare ciascuna soluzione, partendo dai pre-requisiti.
				
		}
		
	}
	
	\newframe{\nameref{sol}}{Concorrenza}{Esempio}{1}{
		\footnotesize
		\myitemize {
			\item Accesso ad un Segmento $S$ da parte di un Treno $T$, dall'estremo $D_T$, $D_T=$\ttt{ First\_End}.
				\myitemize {
					\item \footnotesize Rischio starvation dei Treni in attesa.
				}
			\item Segmento realizzato come risorsa protetta con agente di controllo \ii{monitor}.
			\item $T$ accede solo se:
				\myitemize{
					\item \footnotesize $S$ è libero
					\item $S$ non è libero ma i Treni in transito hanno avuto accesso da $D_T$ e il numero di accessi massimo non è stato raggiunto.
					\item $S$ non è libero ma i Treni in transito hanno avuto accesso da $D_T$, il numero di accessi massimo è stato raggiunto, ma all'estremo opposto non vi sono Treni in attesa.
				}
			\item In tutti gli altri casi $T$ deve attendere presso l'estremo di accesso.
			\item All'uscita da $S$ l'ultimo Treno risveglia i Treni in attesa presso l'estremo opposto.
		}	
	}
	
	\begin{frame}[fragile]
	\frametitle{\newtitle{\nameref{sol}}{Concorrenza}{Esempio}{2}}
	\begin{columns}
	\column{0.65\textwidth}
	\vspace{-1.1cm}
	\tiny\begin{verbatim}
    procedure Access_Segment_Monitor(T:Train,Access_End:Integer) begin
        ...
        if Access_End = First_End then
            while 
                ((not Free) and (Access_End /= Current_Direction)) 
                or
                ((not Free) and (Access_End = Current_Direction) and 
                    (Access_Number = MAX) and (Second_End_Count > 0)) 
            loop
                First_End_Count := First_End_Count + 1;
                wait(Can_Enter_First_End);
                First_End_Count := First_End_Count - 1;
            end loop;
        else
            ... // Simmetrico per accesso dalla direzione opposta
        end if;
        if (Free = True) then
            Free := False;
            if (Access_End /= Current_Direction) then
                Access_Number := 1;
                Current_Direction := Access_End;
            end if;
        else
            if (Access_Number < MAX) then
                Access_Number := Access_Number + 1;
            end if;
        end if;
        ...
    end;
\end{verbatim}
	\column{0.4\textwidth}
		\footnotesize
		\myitemize{
			\item Procedura che regola l'accesso ad un Segmento da parte di un Treno;
			\item Accesso multiplo al Segmento, con numero massimo $MAX$ di ingressi consecutivi per estremo.
			\item \ttt{Current\_Direction} mantiene la direzione dei Treni in transito.
		}
	\end{columns}
	\end{frame}

	
	\newframe{\nameref{sol}}{Concorrenza}{Esempio}{3}{
		\footnotesize
		\cm{0.5}
		\justifying Valutazione dei casi possibili per dimostrare la correttezza della soluzione presentata, una volta che $T$ esegue all'interno della procedure di risorsa protetta \ttt{Access\_Segment\_Monitor}.\\
		\cm{0.3}
		\describe{\footnotesize Caso 1: Accesso Consentito}{
			\footnotesize
			\justifying
			\ii{Precondizione:} \ttt{Free=True}\\
			$T$ imposta il valore di \ttt{Free} a \ttt{False}. Se l'estremo di accesso è diverso da quello corrente, allora il numero di accessi per estremo \ttt{Access\_Number} viene incrementato di $1$, e la direzione corrente \ttt{Current\_Direction} è settata a $1$. In questo modo una volta raggiunto il massimo numero di accessi $MAX$, esso viene re-impostato ad $1$ solo se $T$ proviene da una direzione diversa dall'ultima percorsa. Esegue infine le operazioni previste dopo aver ottenuto l'accesso.
		}
	}
	
	
	\newframe{\nameref{sol}}{Concorrenza}{Esempio}{4}{
		\describe{\footnotesize Caso 2: Accesso Consentito}{
			\footnotesize
			\justifying
			\ii{Precondizione:} \ttt{Free=False} and \ttt{Current\_Direction}$= D_T$ \\and $1<$\ttt{Access\_Number}$<$\ttt{MAX} \\
			Perché sia verificata la Precondizione, almeno un altro Treno proveniente dallo stesso estremo deve aver avuto accesso al Segmento (Caso 1). In questo caso, $T$ si limita a incrementare di $1$ il contatore di accessi per estremo \ttt{Access\_Number}, e ad eseguire le operazioni previste dopo l'accesso.
		}
		\describe{\footnotesize Caso 3: Accesso Consentito}{
			\footnotesize
			\justifying
			\ii{Precondizione:} \ttt{Free=False} and \ttt{Current\_Direction}$= D_T$ and \\\ttt{Access\_Number}$=$\ttt{MAX} and \ttt{Second\_End\_Count}$=0$ \\
			Perché sia verificata la Precondizione, almeno $MAX$ Treni provenienti dallo stesso estremo hanno avuto accesso al Segmento (Caso 1 + Caso 2). In questo caso, $T$ non incrementare il contatore di accessi \ttt{Access\_Number}, ed esegue le operazioni previste dopo l'accesso.
		}
	}
	
	
	\newframe{\nameref{sol}}{Concorrenza}{Esempio}{5}{
		
		\describe{\footnotesize Caso 4: Accesso non consentito}{
			\footnotesize
			\justifying
			\ii{Precondizione:} \ttt{Free=False} and \ttt{Current\_Direction}$\ne D_T$\\
			Perché la pre-condizione sia verificata, almeno un Treno proveniente dall'estremo opposto rispetto a $T$ ha eseguito nel Caso 1. Il thread corrente incrementa il contatore dei Treni in attesa per l'estremo corrente \ttt{First\_End\_Count}, e si pone in attesa su variabile di condizione \\\ttt{Can\_Enter\_First\_End}. 
			
		}
		\describe{\footnotesize Caso 5: Accesso non consentito}{
			\footnotesize
			\justifying
			\ii{Precondizione:} \ttt{Free=False} and \ttt{Current\_Direction}$= D_T$ and \\\ttt{Access\_Number}$=$\ttt{MAX} and \ttt{Second\_End\_Count}$>0$\\
			Perché la pre-condizione sia verificata, almeno un Treno proveniente dall'estremo opposto rispetto a $T$ ha eseguito all'interno di \ttt{Access\_Segment\_Monitor} nel Caso 4 (relativamente alla propria direzione). Il thread corrente incrementa il contatore dei Treni in attesa per l'estremo corrente \ttt{First\_End\_Count}, e si pone in attesa su variabile di condizione \ttt{Can\_Enter\_First\_End}. 
			
		}
	}
	
	
	\newframe{\nameref{sol}}{Tecnologie}{}{}{
		\myitemize {
			\item Scelta delle tecnologie che meglio si adattano alle scelte di progetto.
			\item Interessante l'utilizzo di tecnologie eterogenee
				\myitemize {
					\item \`E difficile pensare ad un Sistema Distribuito realizzato con tecnologie omogenee.
					\item Possibilità di utilizzare tecnologie specifiche per singola componente.\\Ad es. nella soluzione progettata:
					
						\myitemize {
							\item ho utilizzato il linguaggio Ada per codificare le componenti che rappresentano le Regioni;
							\item ho utilizzato il linguaggio Scala per la realizzazione di Name Server, Biglietteria e Controller Centrale;
							\item ho utilizzato Javascript e HTML per la realizzazione dell'interfaccia grafica.
						}
				}
		}
	}
\section{Conclusioni}\label{conclusions}
	
	\newframe{\nameref{conclusions}}{}{}{}{
		\cm{0.5}
		\myitemize {
			\item La progettazione di un Sistema Concorrente e Distribuito è un'operazione complessa.
			\item Molto importante l'analisi iniziale e il confronto fra diverse architetture si Sistema possibili.
			\item Errori comuni:
				\myitemize {
					\item dare per scontato problematiche di distribuzione;
					\item affidarsi a strumenti di linguaggio per risolvere problemi di concorrenza.
				}
		}
	}
	
\end{document}

