\chapter{Il problema}

Il progetto didattico prevede la progettazione e la realizzazione di un simulatore software di un sistema ferroviario. Tale sistema prevede i seguenti requisiti di base (la specifica originale è consultabile all'indirizzo \url{http://www.math.unipd.it/~tullio/SCD/2005/Progetto.html}):
	\begin{itemize}
		\item La presenza di Treni appartenenti a categorie a priorità e modalità di fruizione diverse. 
		\item La presenza di Viaggiatori, che eseguono operazioni elementari come acquisto di un biglietto, salita/discesa su/da un Treno, ecc.
		\item La definizione di un ambiente costituito da un insieme di Stazioni, ciascuna composta da:
			\begin{itemize}
				\item Piattaforme di attesa per Treni e Viaggiatori.
				\item Un Pannello Informativo che visualizza informazioni sui Treni in arrivo, in transito e che hanno appena superato la Stazione corrente.
				\item Una Biglietteria, presso la quale un Viaggiatore può acquistare un biglietto di viaggio.
			\end{itemize}
		\item La presenza di Segmenti di collegamento tra Stazioni, a percorrenza bidirezionale.
		\item La presenza di un entità di controllo globale che mantiene lo stato di ciascun Viaggiatore e ciascun Treno in transito.
		\item L'obbligo da parte di un Viaggiatore di acquistare un Biglietto prima di poter usufruire del servizio ferroviario.
		\item La presenza di collegamenti multipli tra Stazioni, ovvero ciascuna Stazione può essere raggiunta da più Segmenti e da ciascuna stazione possono partire più segmenti.
		\item Il percorso portato a termine da un Viaggiatore può comprendere cambi di treno.
		\item Ciascun Treno appartiene ad una delle seguenti due categorie:
			\begin{description}
				\item {\bb{Regionale}}\\Treno a bassa priorità, senza posto garantito.
				\item {\bb{FB}}\\Treno a priorità più alta, che necessita prenotazione.
			\end{description}
		\item Ciascun Treno possiede una capienza massima di Viaggiatori.
	\end{itemize}
