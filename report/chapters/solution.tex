\chapter{Soluzione}

Di seguito, verrà presentata la soluzione realizzata, in termini di progettazione come sistema distribuito e concorrente.

\section{Architettura Logica di Distribuzione}

Un diagramma infromale delle componenti distribuite che compongono il sistema è presentato in figura X. Di seguito verranno descritte le principali.
	
	\subsection{Regioni}\label{sec:distr_regioni}
	
	La simulazione è stata suddivisa in Regioni, le quali risiederanno su nodi di calcolo diversi. Questa scelta aggiunge i seguenti requisiti minimi alla specifica iniziale:
	\begin{itemize}
		\item I Treni, se previsto dal percorso, possono viaggiare da una Regione all'altra.
		\item I Passeggeri possono raggiungere destinazioni in Regioni diverse.
		\item Esisteranno stazioni che chiameremo di Gateway che permettono a Treni e Passeggeri di raggiungere Regioni diverse. 
		\item Deve essere garantita consistenza temporale nel passaggio da una Regione ad un'altra.
	\end{itemize}
	Da questa scelta consegue inoltre l'introduzione di un semplice Server dei Nomi che mantiene traccia di ciascuna Regione, in modo tale da rendere agevole la risoluzione della locazione alla quale l'entità si trova. 
	
	
	
	\subsection{Biglietterie}
	
	Per poter gestire meglio la definizione di un percorso e l'erogazione di un Biglietto per un Viaggiatore, ho pensato di introdurre una gerarchia su due livelli, di Biglietterie. Ci saranno dunque tre categorie di biglietterie:
		\begin{description}
			\item {\bb{Biglietterie di Stazione}} \\
			Forniscono un'interfaccia adeguata ai Viaggiatori per poter acquistare un biglietto.
			\item {\bb{Biglietterie Regionali}}\\
			Hanno conoscenza regionale della topologia del grafo composto da Stazioni e Segmenti.
			\item {\bb{Biglietteria Cantrale}} \\ 
			Ha conoscenza di più alto livello; in particolare, essa mantiene traccia delle connessioni tra le varie regioni ( ovvero i collegamenti tra Stazioni di Gateway di regioni diverse).
		\end{description} 
	
	\subsection{Controller Centrale}
		
	Il Conterollo Centrale è una entità distribuita, alla quale tutti i nodi inviano Eventi per notificare lo stato di avanzamento globale della simulazione. Esso fornisce una interfaccia alle varie Regioni per ricevere gli Eventi, ed un'interfaccia per permettere a client remoti di poter visualizzare gli effetti di tali Eventi. Quest'ultima possibilità è ottenuta mediante un meccanismo di tipo Publish/Subscribe, attraverso il quale client remoti possono registrarsi presso il Controller per ricevere, in modalità Push, gli Eventi.
	In questo modo è possibile per un qualsiasi client interfacciarsi al Controller e fornire, ad esempio, una rappresentazione grafica della simulazione.

	Il Controller sarà anche responsabile dell'invio di un \ii{segnale di terminazione}, che verrà recapitato a tutti i nodi che componegono la simulazione. 
\newpage
\section{Architettura Logica Concorrente}

Internamente a ciascuna Regione di simulazione, sono definite entità che modellano le interazioni previste dalla specifica del problema. Di seguito sono descritte le principali.

	
	%  ################################################# SEGMENTO ######################################################
	\subsection{Segmento}
	
	Un Segmento (\ttt{Segment}) è modellato come una \ii{entità reattiva ad accesso mutuamente esclusivo, a molteplicità N}, dove \ttt{N} è il numero massimo di utilizzatori, la quale fornisce un'interfaccia utilizzabile da entità attive di tipo Treno per accedere ed uscire in maniera ordinata. Esso è caratterizzato da:
			\begin{itemize}
				\item le due stazioni che esso collega;
				\item una lunghezza;
				\item una velocità massima di percorrenza.
				\item un flag booleano \ttt{Free} che indica lo stato della risorsa, occupata o no.
			\end{itemize}


	%  ################################################# VIAGGIATORE ######################################################
	\subsection{Viaggiatore}\label{subsec:traveler_def}		
	
	In prima analisi, un Viaggiatore (\ttt{Traveler}) può essere modellato come una \ii{entità Attiva} che esegue una sequenza di operazioni semplici.
	La modellazione dell'entità Viaggiatore come Attiva non può però essere semplicemente associata ad un processo, soprattutto in presenza di un modello di distribuzione come quello presentato in sezione \ref{sec:distr_regioni}.
	In questa ipotesi infatti, nel caso in cui un Viaggiatore si spostasse da una Regione ad un'altre, avremmo a disposizione solo due possibili soluzioni:
		\begin{itemize}
			\item La migrazione del processo che rappresenta il Viaggiatore sul nodo (Regioni) di destinazione, come distruzione del processo sul nodo di partenza e creazione dinamica dello stesso sul nodo destinazione. Questa operazione è in generale computazionalmente molto costosa. 
			\item La replicazione del processo che rappresenta il Viaggiatore su tutti i nodi, e l'attivazione, intesa come cambio di stato del processo in modo tale che possa competere per la CPU; tale soluzione è molto costosa in termini di memoria utilizzata, e non scala all'aumentare del numero di passeggeri.
		\end{itemize}
	La soluzione adottata consiste nel disaccoppiare le operazioni svolte da ciascun Viaggiatore dal processo che le esegue, prevedendo una struttura dati costituita da:
		\begin{itemize}
			\item un \ii{pool di M processi} dimensionato in maniera opportuna;
			\item una \ii{coda di operazioni} che man mano vengono estratte ed eseguite dai processi nel pool.
		\end{itemize}
	In questo modo è sufficiente replicare per ciascun Viaggiatore, su tutti i nodi che compongono il sistema, una struttura dati che contiene i suoi dati e le operazioni che esso eseguirà. 
	Il cambio di Regione di un Viaggiatore sarà quindi potrà essere ottenuto semplicemente inserendo la prossima operazione da eseguire per il Viaggiatore nella coda del pool di processi del nodo destinazione.
	
	%  ################################################# TRENO ######################################################
	\subsection{Treno}
	Un Treno (\ttt{Train}) è una \ii{entità Attiva}, la quale esegue ciclicamente un numero finito di operazioni. Ciascuna entità Treno può appartenere a due categorie, FB a priorità più alta, e Regionale a priorità minore, ed è caratterizzata da:
		\begin{itemize}
			\item un identificativo univoco;
			\item una capienza massima;
			\item una velocità massima raggiungibile;
			\item una velocità corrente;
		\end{itemize}
	
	Anche in questo caso come per le entità di tipo Viaggiatore, ciascuna entità non viene mappata su un singolo processo. Infatti anche per le entità di tipo Treno ciò comporterebbe una complessa e costosa, in termini computazionali e di memoria, gestione del passaggio da un nodo (Regione) ad un altro. Ho optato invece nel progettare una struttura dati che mantiene:
		\begin{itemize}
			\item una coda di Descrittori di Treno, che contengono tutte le informazioni che distinguono una entità Treno;
			\item un pool di processi a dimensionato in maniera opportuna; ciascun processo sarà tale per cui una volta ottenuto un descrittore dalla coda, eseguirà per lui una fissata sequenza di azioni.
		\end{itemize}
	In questo modo un entità Treno eseguirà all'interno di un nodo se e soltanto se il suo descrittore sarà inserito nella coda di descrittori della struttura dati descritta.
				
	%  ################################################# STAZIONE ######################################################
	\subsection{Stazione}
	
	Una Stazione (\ttt{Station}) è modellata come una struttura dati contenete:
		\begin{itemize}
			\item un certo numero $ n > 2$ di Piattaforme (\ttt{Platform});
			\item una Biglietteria (\ttt{Ticket\_Office});
			\item un Pannello Informativo (\ttt{Notice\_Panel}).
		\end{itemize}
	Essa offre una interfaccia alle entità Treno e Viaggiatore per l'accesso a Piattaforme e Biglietteria.
		
		\subsubsection{Piattaforma}
	
		Una Piattaforma è modellata come una \ii{entità reattiva ad accesso mutuamente esclusivo, a molteplicità 1}. Essa espone un'interfaccia che permette alle entità Treno di potervi sostare ed effettuare discesa e salita delle entità Passeggero o di poter superare la stazione, e alle entità Passeggero di accodarsi in attesa di uno specifico Treno.
		
		\subsubsection{Biglietteria}
		
		Una Piattaforma è modellata come una interfaccia che permette al Viaggiatore di acquisire un Biglietto di viaggio. 
		
		\subsubsection{Pannello Informativo}
		
		Una Piattaforma è modellata come una \ii{entità reattiva con agente di controllo, a molteplicità 1}. Esso espone una interfaccia tale da permettere alla stazione di notificare lo stato delle entità Treno che stanno arrivando, quelle in sosta e quelle in partenza.

\section{Interazione tra le Entità}

	\subsection{Viaggio di un Viaggiatore}
	
	Alla luce della definizione dell'entità Viaggiatore in sezione \ref{subsec:traveler_def}, è possibile definire le azioni che compongono il viaggio:
		\begin{itemize}
			\item Acquisto di un Biglietto (\ttt{Ticket}) presso la Biglietteria della Stazione (\ttt{Ticket\_Office}) di partenza. Ciascun Biglietto è composto da una sequenza ordinata di Tappe (\ttt{Ticket\_Stages}), ciascuna contenente:
				\begin{verbatim}
					- start_station
					- next_station
					- train_id 
					- start_platform 
					- destination_platform
				\end{verbatim}
			\item Una volta ottenuto un Ticket, vengono eseguite le seguenti operazioni per ciascuna Tappa del Biglietto:
				\begin{itemize}
					\item Accodamento presso la Piattaforma \ttt{start\_platform} della Stazione \ttt{start\_station} in attesa del Treno \ttt{train\_id}.
					\item All'arrivo del treno \ttt{train\_id}, Accodamento presso la Piattaforma \ttt{destination\_platform} della Stazione \ttt{next\_station} in attesa dell'arrivo di \ttt{train\_id}. 
				\end{itemize}
		\end{itemize} 
	Le tre operazioni sopraelencate possono essere incapsulate in strutture dati che rappresentano una specifica operazione. In questo modo viene generata una gerarchia di operazioni, tutte derivanti da un'unica operazione generica.
	
	\subsection{Viaggio di un Treno}
	
	A ciascuna entità Treno, è assegnato un Percorso (\ttt{Route}) di andata e un Percorso di ritorno, ovvero sequenze di Tappe (\ttt{Stage}) successive, ciascuna composta dalla n-upla:
				\begin{center}
					\begin{verbatim}
						       <next_segment,next_station,next_platfom,action>
					\end{verbatim}
				\end{center}
dove \ttt{action} indica quello che un Treno dovrà compiere presso la prossima stazione, a scelta tra \ttt{ENTER}, per entrare ed effettuare discesa e salita passeggeri o \ttt{PASS} per non fermarsi e oltrepassare la Stazione.

Le operazioni effettuate sono, per ciascuna Tappa del Percorso corrente (di andata o di ritorno):
				\begin{itemize}
					\item Accesso al prossimo Segmento \ttt{next\_segment} previsto.
					\item Percorrenza all'interno del Segmento come attesa finita di durata proporzionale alla lunghezza del Segmento e alla velocità massima alla quale il Treno può percorrerlo.
					\item Uscita dal Segmento e richiesta di Accesso alla Stazione successiva (\ttt{next\_station}) presso la Piattaforma indicata da \ttt{next\_platfom}, per eseguire l'azione \ttt{action}.
					\item Se \ttt{action = ENTER} allore effettua discesa e salita dei Viaggiatori in attesa dell'arrivo del Treno.
					\item Uscita dalla Piattaforma \ttt{next\_platfom}.
				\end{itemize}
	
		\subsubsection{Accesso al prossimo Segmento}
		
		L'accesso alla risorsa protetta Segmento è regolato da una interfaccia ben definita, che permette:
			\begin{itemize}
				\item Ingresso.
				\item Uscita.
			\end{itemize}
		Per mantenere l'ordine di accesso, ciascun Segmento è dotato di una Coda FIFO (\ttt{Train\_Queue}) che conterrà i Descrittori dei Treni correntemente in transito.
		\begin{description}
			\item {\bb{Ingresso}} \\
			La richiesta di accesso al segmento avviene in mutua esclusione tra tutti le entità Treno. In ogni momento quindi solo una entità eseguirà all'interno della risorsa protetta. Una volta ottenuta la risorsa ciascun Treno compie le seguenti operazioni
			 \begin{itemize}
			 	\item Inserimento del Descrittore nella coda \ttt{Train\_Queue};
			 	\item Aggiornamento della velocità di percorrenza del Treno entrato in base a quella dei treni che lo precedono.
			 	\item Nel caso in cui la risorsa risulti vuota (viene consultato il flag booleano che mantiene questa informazione), allora viene modificato il valore del flag in modo tale da indicare lo stato occupato della risorsa, e memorizzata la stazione di provenienza del Treno.
			\end{itemize}
			Una volta terminate queste due operazioni, il Treno rilascia la risorsa e, basandosi sulla lunghezza e sulla velocità da mantenere, simula la percorrenza rendendosi inattivo (non competitivo per l'ottenimento della CPU) per un tempo dato dalla semplice equazione: $ Time = Segment\_Length / Actual\_Speed $.
			
			Nel caso in cui una volta ottenuta la risorsa protetta Segmento, un Treno richiedesse l'accesso nella direzione opposta a quella dei Treni che percorrono il Segmento (ovvero abbiamo la situazione in cui il flag booleano indica lo stato occupato, e la stazione di provenienza memorizzata presso il Segmento è diversa da quella del Treno richiedente) allora questo dovrà attendere fino a che il Segmento non sarà si sarà liberato dai treni in transito.
			
			Una prima soluzione possibile è quella di prevedere una coda di attesa interna alla risorsa Segmento (\ttt{Waiting\_Queue}), per i Treni provenienti dalla direzione opposta rispetto al senso di marcia corrente: tali Treni dovranno avere priorità maggiore nell'accedere al Segmento appena esso diventa vuoto, rispetto ad altri Treni che sopraggiungono successivamente.
			
			Questa prima modellazione, non risulta però sufficiente a garantire che i Treni nella coda avranno accesso al Segmento in qualche momento: infatti vi è il rischio di portare a Starvation dei Treni in attesa. Si consideri la presenza di un percorso circolare composto da N Segmenti, e di M Treni che viaggiano lungo tale percorso in senso orario, in modo tale che in ogni istante ci sia almeno un treno all'interno di ciascun Segmento. Se ora aggiungiamo al sistema un Treno che viaggia in direzione opposta, allora esso, adottando la semantica di accesso descritta non riuscirà mai a percorrere uno dei Segmenti.			
			
			 %Questo problema può essere evitato aggiungendo un parametro al Segmento che determina il massimo numero di Treni che, da ciascuna direzione, possono circolare su di esso. Una volta che esso è stato raggiunto, i Treni che dovessero sopraggiungere con lo stesso senso di marcia rispetto ai treni in transito verranno accodati internamente alla risorsa, con priorità minore di accedere successivamente rispetto ai treni nella coda \ttt{Waiting\_Queue}, ma maggiore rispetto ai Treni che richiedono l'accesso dall'esterno.
			
			Una possibile soluzione al problema, è mantenere nella coda \ttt{Waiting\_Queue} tutte le entità Treno che non possono accedere correntemente, sia perché provenienti dalla direzione opposta, sia perché il numero massimo di accessi da una direzione è stato raggiunto. In questo modo la semantica di accesso viene modificata come segue:
				\begin{itemize}
					\item Se \ttt{Free = True} allora il Treno corrente 
						\begin{itemize}
							\item Modifica il valore di \ttt{Free} a \ttt{False}.
							\item Imposta la Stazione provenienza con la propria.
							\item Incrementa di 1 il contatore di accessi.
						\end{itemize}
					\item Se invece \ttt{Free = False} allora
						\begin{itemize}
							\item Se la stazione di provenienza del Treno corrente è diversa da quella memorizzata nel Segmento, allora inserisci il Treno nella coda \ttt{Waiting\_Queue} e STOP.
							\item Altrimenti 
								\begin{itemize}
									\item Se il contatore di accessi è $ < $ del limite massimo incrementa di 1 il contatore di accessi.
									\item Altrimenti accoda il Treno su \ttt{Waiting\_Queue} e STOP.
								\end{itemize}
						\end{itemize}
					\item Se il Treno ha avuto accesso al Segmento, inserisci il Descrittore del Treno nella coda \ttt{Train\_Queue} e modifica velocità di transito del Treno a seconda della massima velocità possibile.
				\end{itemize}
			
			
		\end {description}
	
	
	

