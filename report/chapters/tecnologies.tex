\chapter{Tecnologie Adottate}

La scelta delle tecnologie per l'implementazione del prototipo realizzato, è stata operata nell'ottica di individuare strumenti che permettessero una agevole attuazione della soluzione descritta. 
Di seguito verranno presentati i linguaggi di programmazione adottati e le tecnologie scelte.

	\section{Scala}
		Il linguaggio Scala è un linguaggio funzionale \ii{Object Oriented}. Ho utilizzato questo linguaggio per la realizzazione di 
			\begin{itemize}
				\item \bb{Controller Centrale}
				\item \bb{Biglietteria Centrale}
				\item \bb{Application Server}, il quale invia gli eventi di simulazione in modalità push ad un client realizzato in linguaggio HTML e Javascript, mediante protocollo \ii{Web Socket}, per poter fornire una rappresentazione grafica. Per la sua realizzazione è stato utilizzato il framework MVC \ttt{Play 2.0} (disponibile all'indirizzo \url{http://www.playframework.com/}).
				\item \bb{Name Server} per la gestione delle Regioni di simulazione.
			\end{itemize}
		Internamente alle componenti distribuite, la concorrenza è stata gestita adottando il modello di concorrenza ad \ii{Attori} fornito nativamente dal linguaggio. 
		
	\section{Ada}
		Il linguaggio Ada è stato utilizzato per realizzare la componente di simulazione di ciascuna Regione. La scelta di questa tecnologia è stata guidata dal modello di concorrenza che essa offre, che mi è sembrato più adatto per rappresentare le interazioni tra le entità introdotte al capitolo precedente rispetto ai modelli offerti da altri linguaggi. 
		In prima battuta, ho valutato l'utilizzo del modello di concorrenza ad \ii{Attori} di Scala; tale modello avrebbe garantito una separazione naturale tra Attore e Thread che lo esegue, e quindi fornito un meccanismo di linguaggio scalabile per l'esecuzione delle entità Viaggiatore e Treno, da utilizzare in sostituzione a quello descritto nelle sezioni \ref{subsec:traveler_def} e \ref{subsec:train_def}. Tuttavia, il modello ad \ii{Attori} avrebbe reso necessaria l'adozione di un \ii{agente di controllo Server} a protezione delle entità reattive come Segmenti e Piattaforme, e ciò avrebbe comportato:
			\begin{itemize}
				\item la necessità di utilizzare thread per l'esecuzione degli \ii{Attori} a protezione delle entità;
				\item una maggiore complessità di terminazione;
			\end{itemize}
		L'utilizzo di Ada ha permesso un buon livello di controllo di allocazione di thread, risorse protette e oggetti, e il suo sistema di tipi molto restrittivo ha consentito di ridurre i possibili errori in fase di sviluppo.
	
	\section{Yami4}
L'interazione tra componenti remote del sistema, realizzate con tecnologie eterogenee, è stata possibile utilizzando il middleware per lo scambio di messaggi \ii{Yami4} (\url{http://www.inspirel.com/yami4/}), compatibile, tra gli altri linguaggi, con Ada e Java (e quindi di conseguenza anche con Scala). Questo strumento è stato preferito agli altri possibili middleware (\ii{Distributed Systems Annex} per RPC in Ada e \ii{CORBA}) in quanto si è rivelato molto semplice da utilizzare e versatile, e ha permesso di adottare un certo grado di disaccoppiamento tecnologico tra le componenti. 

Per ciascun Nodo di simulazione del Sistema è stato utilizzato un singolo \ii{Agente} (oggetto fornito da \ii{Yami4} che permette l'invio e la ricezione di messaggi remoti), sia per quanto riguarda l'invio che la ricezione di messaggi remoti, in ascolto su una singola porta (adottando il protocollo TCP). Per ciascun \ii{Agente}, è possibile infatti definire:
	\begin{itemize}
		\item \ttt{Object}: un oggetto remoto raggiungibile ad un dato indirizzo, identificato univocamente da una stringa.
		\item \ttt{Service}: insieme di servizi offerti da un oggetto remoto, identificati univocamente da una stringa.
	\end{itemize}
Per quanto riguarda i nodi di simulazione, ho utilizzato quindi un unico \ttt{Object}, il quale mette a disposizione un \ttt{Service} per ciascun tipo di servizio richiesto. Questa soluzione mi è sembrata la più semplice e estendibile possibile.
Il protocollo per la serializzazione dei dati utilizzato per lo scambio di messaggi, è quello fornito da \ii{Yami4}, e quindi lo scambio di dati in formato stringa chiave-valore; inoltre per l'invio di strutture dati complesse, come Descrittori di Treni o Passeggeri, e Biglietti, ho utilizzo lo standard \ii{JSON}. Per la codifica e decodifica di stringhe \ii{JSON} mi sono servito di due librerie:
	\begin{itemize}
		\item La libreria \ii{Gnatcoll} fornita con la distribuzione \ttt{Gnat}, per il linguaggio Ada.
		\item La libreria \ii{json-smart} (disponibile all'indirizzo \url{https://code.google.com/p/json-smart/}), scritta in Java e integrata nelle componenti mediante \ii{wrapper} Scala.
		
	\end{itemize}
	Il formato \ii{JSON} è utilizzato anche per la definizione dei file di configurazione che descrivono le entità del sistema.



