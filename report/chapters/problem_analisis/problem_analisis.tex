\chapter{Analisi del problema}

La progettazione di un simulatore di un sistema ferroviario presenta diverse problematiche relativamente alla concorrenza e alla distribuzione, in quanto:
\begin{itemize}
	\item il problema prevede l'interazione tra una popolazione di entità in maniera concorrente;
	\item vi sono dei punti di sicronizzazione tra le entità, che devono essere identificati e modellati opportunamente;
	\item non è ragionevole fare assunzioni a priori sulle tecnologie che risolveranno il problema, ne sull'ambiente di esecuzione;
	\item è ragionevole prevedere un certo livello di distribuzione delle componenti del sistema;
	\item si possono presentare difficoltà dovute ai ritradi nella trasmissioni di rete tra le componenti.
\end{itemize} 

Di seguito andrò ad analizzare quelle che sono le principali. 

% ########################################### DISTRIBUZIONE ##################################################
\section{Distribuzione}

Le problematiche legate alla distribuzione sono molteplici. Nel progetto di un simulatore di un sistema ferroviario infatti, la presenza di entità distribuite è auspicabile, sia per suddividere logicamente le entità, sia per distribuire l'onere di calcolo su nodi differenti.
Le caratteristiche desiderabili da un sistema dostribuito che simula una struttura ferroviaria sono:
	\begin{itemize}
		\item Il complesso deve apparire all'utente come un sistema unitario, la natura distibuita del sistema deve essere nascosta all'utilizzatore finale. 
		\item L'architettura distribuita non deve limitare le funzionalità desiderate.
		\item \'E desiderabile che vi sia un buon grado di disaccoppiamento tra le componenti, e dalle tecnologie adottate per la comunicazione tra nodi della rete.
		\item Il sistema dovrà essere il più possibile robusto agli errori.
		\item La progettazione architetturale deve prevedere un meccanismo che permetta Avvio del sistema e Terminazione ordinata.
		\item L'architettura distribuita deve sottostare a vincoli temporali propri di un sistema ferroviario.
		\item La struttura distribuita deve essere tale da permettere estendibilità e scalabilità.
	\end{itemize}

	\subsection{Gestione del Tempo}
	
	La simulazione è scandita da orari di partenza e di arrivo dei Treni che circolano tra le stazioni. Per questo è importante dotare il sistema di un \ii{riferimento temporale} adeguato, che permetta di gestire i ritardi introdotti dalla comunicazione di rete, o dalla diversità di sincronizzazione degli orologi dei diversi nodi della rete.
	 
	Tale problema assume forme diverse in base al grado di distribuzione scelto per le componenti che generano gli eventi caratterizzanti la simulazione. In particolare, un livello di distribuzione alto, che prevede ad esempio la collocazione di una entità Stazione per nodo della rete, richiederà un meccanismo di regolazione del tempo più complesso e delicato rispetto ad un sistema con un livello di distribuzione più contenuto, che preveda ad esempio una distribuzione di singole Regioni di simulazione. 
	
	La scelta del riferimento temporale diviene quindi cruciale per lo svolgersi della simulazione. Abbiamo due tipi possibili di orologio:
		\begin{itemize}
			\item Assoluto: Prevede l'esistenza di un'entità dalla quale le varie componenti attingono per ottenere l'informazione temporale. 
			\item Relativo: Ciascun nodo di calcolo possiede un proprio riferimento temporale interno.
		\end{itemize}
		
	\'E chiaro che la prima soluzione non si presta ad essere utilizzata per il problema presentato. Infatti esso, possedendo un flusso coninuo interno del tempo, non permetterebbe a entità indipendenti su nodi diversi di eseguire logicamente allo stasso istante (ad esempio due treni che in nodi diversi partono contemporaneamente da una stazione)

	\subsection{Acquisto di un Biglietto}
	
	In base al grado di distribuzione della modellazione realizzata, è necessario prevedere una struttura distribuita di biglietterie, in quanto la natura del problema prevede un livello di conoscenza globale soprattutto per l'acquisto di biglietti di treni a prenotazione.

	\subsection{Terminazione del Sistema}
	
	La durata di una simulazione di un sistema ferroviario è per sua natura indefinita. \'E quindi necessario un intervento esterno che ne decreti la terminazione.	La \ii{Terminazione del sistema} globale deve essere coordinata tra tutte le componenti distribuite, ed effettuata in modo tale da non far terminare l'esecuzione in uno stato inconsistente. Dovrà inoltre garantire che nessun nodo di calcolo rimarrà attivo (ad esempio, nessun thread in esecuzione o in attesa).

	\subsection{Avvio del Sistema}
	
	L'\ii{Avvio del sistema} dove essere progettato in modo tale da permettere a tutte le componenti distribuite di interagire, ed evitare errori. In particolare:
	\begin{itemize}
		\item dev'essere previsto un meccanismo che permetta una rapida individuazione dei nodi con i quali ciascuna entità coopera;
		\item devono essere evitati (o gestiti) errori causati dal tentativo di comunicazione di thread concorrenti con entità non ancora pronte o allocate.
	\end{itemize} 
	


\section{Logica di Concorrenza}

Internamente a ciascuna Regione di simulazione, sono definite entità che modellano le interazioni previste dalla specifica del problema. Di seguito sono descritte le principali.

	
	%  ################################################ SEGMENTO ######################################################
	\subsection{Segmento}
	
	Un Segmento (\ttt{Segment}) è modellato come una \ii{entità reattiva ad accesso mutuamente esclusivo, a molteplicità N}, dove \ttt{N} è il numero massimo di utilizzatori, la quale fornisce un'interfaccia utilizzabile da entità attive di tipo Treno per accedere ed uscire in maniera ordinata. Esso è caratterizzato da:
			\begin{itemize}
				\item le due stazioni che esso collega;
				\item una lunghezza;
				\item una velocità massima di percorrenza.
				\item un flag booleano \ttt{Free} che indica lo stato della risorsa, occupata o no.
			\end{itemize}


	%  ################################################# VIAGGIATORE ######################################################
	\subsection{Viaggiatore}\label{subsec:traveler_def}		
	
	In prima analisi, un Viaggiatore (\ttt{Traveler}) può essere modellato come una \ii{entità Attiva} che esegue una sequenza di operazioni semplici.
	La modellazione dell'entità Viaggiatore come Attiva non può però essere semplicemente associata ad un processo, soprattutto in presenza di un modello di distribuzione come quello presentato in sezione \ref{sec:distr_regioni}.
	In questa ipotesi infatti, nel caso in cui un Viaggiatore si spostasse da una Regione ad un'altre, avremmo a disposizione solo due possibili soluzioni:
		\begin{itemize}
			\item La migrazione del processo che rappresenta il Viaggiatore sul nodo (Regioni) di destinazione, come distruzione del processo sul nodo di partenza e creazione dinamica dello stesso sul nodo destinazione. Questa operazione è in generale computazionalmente molto costosa. 
			\item La replicazione del processo che rappresenta il Viaggiatore su tutti i nodi, e l'attivazione, intesa come cambio di stato del processo in modo tale che possa competere per la CPU; tale soluzione è molto costosa in termini di memoria utilizzata, e non scala all'aumentare del numero di passeggeri.
		\end{itemize}
	La soluzione adottata consiste nel disaccoppiare le operazioni svolte da ciascun Viaggiatore dal processo che le esegue, prevedendo una struttura dati (\ttt{Traveler\_Pool}) costituita da:
		\begin{itemize}
			\item un \ii{pool di M processi} dimensionato in maniera opportuna;
			\item una \ii{coda di operazioni} che man mano vengono estratte ed eseguite dai processi nel pool.
		\end{itemize}
	In questo modo è sufficiente replicare per ciascun Viaggiatore, su tutti i nodi che compongono il sistema, una struttura dati che contiene i suoi dati e le operazioni che esso eseguirà, e cioè un Descrittore (\ttt{Traveler\_Descriptor}). 
	Il cambio di Regione di un Viaggiatore sarà quindi potrà essere ottenuto semplicemente inserendo la prossima operazione da eseguire per il Viaggiatore nella coda del pool di processi del nodo destinazione.
	
	%  ################################################# TRENO ######################################################
	\subsection{Treno} \label{subsec:train_def}
	Un Treno (\ttt{Train}) è una \ii{entità Attiva}, la quale esegue ciclicamente un numero finito di operazioni. Ciascuna entità Treno può appartenere a due categorie, FB a priorità più alta, e Regionale a priorità minore, ed è caratterizzata da:
		\begin{itemize}
			\item un identificativo univoco;
			\item una capienza massima;
			\item una velocità massima raggiungibile;
			\item una velocità corrente;
		\end{itemize}
	
	Anche in questo caso come per le entità di tipo Viaggiatore, ciascuna entità non viene mappata su un singolo processo. Infatti anche per le entità di tipo Treno ciò comporterebbe una complessa e costosa, in termini computazionali e di memoria, gestione del passaggio da un nodo (Regione) ad un altro. Ho optato invece nel progettare una struttura dati \ttt{Train\_Pool} che mantiene:
		\begin{itemize}
			\item una coda di Descrittori di Treno, che contengono tutte le informazioni che distinguono una entità Treno;
			\item un pool di processi a dimensionato in maniera opportuna; ciascun processo sarà tale per cui una volta ottenuto un descrittore dalla coda, eseguirà per lui una fissata sequenza di azioni.
		\end{itemize}
	In questo modo un entità Treno eseguirà all'interno di un nodo se e soltanto se il suo descrittore sarà inserito nella coda di descrittori della struttura dati descritta.
				
	%  ################################################# STAZIONE ###################################################
	\subsection{Stazione} \label{subsec:station}
	
	Una Stazione (\ttt{Station}) è modellata come una struttura dati contenete:
		\begin{itemize}
			\item un certo numero $ n > 2$ di Piattaforme (\ttt{Platform});
			\item una Biglietteria (\ttt{Ticket\_Office});
			\item un Pannello Informativo (\ttt{Notice\_Panel}).
		\end{itemize}
	Essa offre una interfaccia alle entità Treno e Viaggiatore per l'accesso a Piattaforme e Biglietteria.
		
		\subsubsection{Piattaforma}
	
		Una Piattaforma è modellata come una \ii{entità reattiva ad accesso mutuamente esclusivo, a molteplicità 1}. Essa espone un'interfaccia che permette alle entità Treno di potervi sostare ed effettuare discesa e salita delle entità Viaggiatore o di poter superare la stazione, e alle entità Viaggiatore di accodarsi in attesa di uno specifico Treno.
		Internamente essa mantiene due code di \ttt{Traveler\_Descriptor}:
		\begin{itemize}
			\item \ttt{Leaving\_Queue}, che conterrà i descrittori dei Viaggiatori in attesa di Treni per poter effettuare la partenza.
			\item \ttt{Arrival\_Queue}, che conterrà i descrittori dei Viaggiatori in attesa di Treni per poter effettuare l'arrivo alla stazione.
		\end{itemize}
		Inoltre contiene un flag booleano \ttt{Free} che ne indica lo stato di occupazione.
		Ciascuna Piattaforma, in base al tipo di Stazione avrà percorrenza bidirezionale o uniderezionale.
				
		\subsubsection{Biglietteria}
		
		Una Piattaforma è modellata come una interfaccia che permette al Viaggiatore di acquisire un Biglietto di viaggio. 
		
		\subsubsection{Pannello Informativo}
		
		Una Piattaforma è modellata come una \ii{entità reattiva con agente di controllo, a molteplicità 1}. Esso espone una interfaccia tale da permettere alla stazione di notificare lo stato delle entità Treno che stanno arrivando, quelle in sosta e quelle in partenza.
\newpage

