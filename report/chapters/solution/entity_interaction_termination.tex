\newpage
\subsection{Terminazione distribuita}\label{sec:distributed_termination}

La Terminazione del Sistema progettata, sfrutta la presenza dell'entità di Controllo Centrale, la quale, per ciascun nodo di simulazione, invia un messaggio di terminazione. Internamente a ciascun nodo, la terminazione può avvenire semplicemente inviando un messaggio di terminazione ai pool di thread \ttt{Train\_Pool} e \ttt{Traveler\_Pool}, in modo tale che ciascun thread in attesa sulla rispettiva coda di Trenoi o Operazioni da eseguire, interrompa tale attesa e concluda la propria esecuzione. Una soluzione che si limiti a terminare singiolarmente tutti i nodi di esecuzione, tuttavia questo non è sufficiente ne a garantire che tutti i thread terminino la propria esecuzione, ne che lo stato risultante sia consistente. In primo luogo infatti, ci possono essere thread in attesa su una guardia booleana, la quale può non essere più riaperta a causa della terminazione del thread incaricato. \'E il caso dell'attesa per l'accesso ad una Piattaforma da parte di un Treno: infatti nel momento in cui il thread rappresentante il Treno che occupa la Piattaforma termina la propria esecuzione, nessun'altra entità provvederà ad aprire la guardia booleana per permettere l'accesso ai thread in attesa. Nonostante lo stato risulti consistente (un Treno occupa una Piattaforma, altri Treni sono in attesa), il sistema non può considerarsi terminato. Risulta quindi opportuno prevedere un meccanismo capace di interrompere l'attesa su guardia, discriminando il caso in cui essa avvenga per terminazione. In tutti gli altri casi in cui si può avere attesa (accesso ad un Segmento), la semantica descritta nelle sezioni precedenti garantisce che i Treni completeranno le operazioni in modo tale da non rimanere bloccati su guardie booleane (evitando così situazioni di Stallo).

	In secondo luogo, ci possono essere problemi legati alle richieste di natura asincrona come la creazione di Ticket, la quale coinvolge la Biglietteria Centrale e i nodi di simulazione: ad esempio, se supponiamo che i nodi di simulazione abbiano ricevuto un messaggio di terminazione e che alcuni di essi abbiano già terminato la propria esecuzione, la Biglietteria Centrale sarà impossibilitata a contattarli, generando un messaggio di errore sia in fase di creazione, sia in fase di consegna del Ticket creato. Questo può portare a situazioni non consistenti, come per esempio la situazione in cui un Viaggiatore in attesa (non attiva poiché il processo è asincrono, e quindi teminabile) di un Ticket non lo riceve per un errore della Biglietteria Centrale nella costruzione del Biglietto a causa dell'impossibilità di contattare nodi già terminati.
	
	Per ovviare a problemi di questo tipo, è opportuno che il processo di terminazione coinvolga anche la Biglietteria Centrale. Una possibile soluzione al problema della terminazione è la seguente: definiamo \ttt{Marker} un messaggio speciale utilizzato per indicare l'inizio della procedura di terminazione. 
	\begin{itemize}
		\item Il Controller Centrale inizia la procedura di terminazione e invia un messaggio \ttt{Marker} alla Biglietteria Centrale.
		\item Una volta ricevuto tale messaggio, la Biglietteria Centrale inizia a memorizzare in una coda FIFO tutti i messaggi di richiesta (asincrona) di Creazione di un Biglietto. Essa inoltre, completa le richieste di creazione di un Ticket in corso e memorizza i messaggi in uscita relativi a tali richieste su una coda di output.
		Infine, la Biglietteria Centrale invia un messaggio di conferma di terminazione delle operazioni asincrone correnti al Controller Centrale.
		\item Il Controller invia quindi un \ttt{Marker} a tutti i nodi di simulazione, i quali:
			\begin{itemize}
				\item ultimano le operazioni in corso;
				\item inviano un messaggio di \ttt{Unbind} al Name Server;
				\item inviano un messaggio di risposta al Controller Centrale;
				\item infine terminano.
			\end{itemize}
		\item Il Controller una volta ottenuta risposta da tutti i nodi di simulazione, invia un nuovo \ttt{Marker} alla Biglietteria Centrale, la quale, memorizzato lo stato delle code di input e output, termina.
		\item Il Controller Centrale termina la propria esecuzione.
	\end{itemize}
	
	La soluzione presentata permette quindi la terminazione delle entità distribuite, in uno stato consistente. In questo modo memorizzando tale stato, è possibile realizzare una procedura di ripristino dell'esecuzione. 
