\subsection{Percorrenza di un Viaggiatore}
	
	Alla luce della definizione dell'entità Viaggiatore in sezione \ref{subsec:traveler_def}, è possibile definire le azioni che compongono il viaggio:
		\begin{itemize}
			\item Acquisto di un Biglietto (\ttt{Ticket}) presso la Biglietteria della Stazione (\ttt{Ticket\_Office}) di partenza. Ciascun Biglietto è composto da una sequenza ordinata di Tappe (\ttt{Ticket\_Stages}), ciascuna contenente:
				\begin{verbatim}
					- start_station
					- next_station
					- train_id 
					- start_platform 
					- destination_platform
				\end{verbatim}
			\item Una volta ottenuto un Ticket, vengono eseguite le seguenti operazioni per ciascuna Tappa del Biglietto:
				\begin{itemize}
					\item Accodamento presso la Piattaforma \ttt{start\_platform} della Stazione \ttt{start\_station} in attesa del Treno \ttt{train\_id}.
					\item All'arrivo del treno \ttt{train\_id}, Accodamento presso la Piattaforma \ttt{destination\_platform} della Stazione \ttt{next\_station} in attesa dell'arrivo di \ttt{train\_id}. 
				\end{itemize}
		\end{itemize} 
	Le tre operazioni sopraelencate possono essere incapsulate in strutture dati che rappresentano una specifica operazione. In questo modo viene generata una gerarchia di operazioni, tutte derivanti da un'unica operazione generica.

	
