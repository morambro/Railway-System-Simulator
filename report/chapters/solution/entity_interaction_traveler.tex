\subsection{Percorrenza di un Viaggiatore}
	
	Alla luce della definizione dell'entità Viaggiatore in sezione \ref{subsec:traveler_def}, è possibile definire le azioni che compongono il viaggio:
		\begin{itemize}
			\item Acquisto di un Biglietto (\ttt{Ticket}) presso la Biglietteria della Stazione (\ttt{Ticket\_Office}) di partenza. Ogni Biglietto è composto da una sequenza ordinata di Tappe (\ttt{Ticket\_Stages}), ciascuna contenente:
				\begin{verbatim}
					- start_station
					- next_station
					- train_id 
					- start_platform 
					- destination_platform
				\end{verbatim}
			e da un indice della prossima tappa del Percorso (\ttt{Next\_Stage}).
			
			\item Una volta ottenuto un Ticket, vengono eseguite le seguenti operazioni per ciascuna Tappa del Biglietto:
				\begin{itemize}
					\item Accodamento presso la Piattaforma \ttt{start\_platform} della Stazione \ttt{start\_station} in attesa del Treno \ttt{train\_id}.
					\item All'arrivo del treno \ttt{train\_id}, Accodamento presso la Piattaforma \ttt{destination\_platform} della Stazione \ttt{next\_station} in attesa dell'arrivo di \ttt{train\_id}. 
				\end{itemize}
		\end{itemize} 
	Le azioni sopraelencate possono essere incapsulate in strutture dati che rappresentano una specifica operazione. In questo modo viene generata una \ii{gerarchia di operazioni}, tutte derivanti da un'unica \ii{operazione generica}. Di seguito identificheremo tali operazioni rispettivamente con \ttt{BUY\_TICKET}, \ttt{LEAVE} e \ttt{ARRIVE}.
	
	Il protocollo di operazioni che vengono eseguite da un Viaggiatore, è stato mantenuto il più semplice possibile, e l'intero percorso viene regolato dagli eventi generati dalle entità Treno alla partenza e all'arrivo dalle/nelle Stazioni. 
	
	\subsubsection{Acquisto di un Biglietto}
	
	TODO
	...
	\\
	\\
	
	L'ultima azione eseguita dall'Operazione di acquisto di un biglietto consiste nel caricare nella coda della struttura dati \ttt{Traveler\_Pool} descritta nella sezione \ref{subsec:traveler_def}, l'operazione \ttt{LEAVE} del Viaggiatore corrente.
	
	\subsubsection{Partenza da una Stazione}
	
	Denominiamo $t$ la tappa di indice \ttt{Next\_Stage}, ovvero la tappa corrente. Perché l'operazione di Partenza dalla stazione corrente venga effettuata, è necessaria la precondizione per la quale:
	\begin{itemize}
		\item l'operazione \ttt{LEAVE} è stata inserita nella apposita coda di \ttt{Traveler\_Pool};
		\item in qualche momento essa venga prelevata (rimossa) da tale luogo ed eseguita da uno dei thread nel Pool della struttura. 
	\end{itemize}
Le azioni compiute dall'operazione \ttt{LEAVE} sono:
	
	\begin{itemize}
		\item Estrazione \ttt{start\_station} da $t$;
		\item Tramite l'interfaccia esposta dalla Stazione \ttt{start\_station} (una procedura che identifichiamo con \ttt{Wait\_For\_Train\_To\_Go}), viene richiesto di attendere presso la Piattaforma \ttt{start\_platform}.
	\end{itemize} 
	
	Internamente alla Stazione, tale richiesta viene tradotta nell'inserimento del Viaggiatore, o meglio del suo Descrittore \ttt{Traveler\_Descriptor}, nella coda \ttt{Leaving\_Queue} della Piattaforma indicata da \ttt{start\_platform}.
	
	\subsubsection{Arrivo alla Destinazione successiva}
		
	Similmente a quanto descritto per la partenza, l'arrivo a destinazione prevede che l'operazione \ttt{ARRIVE} sia già stata inserita all'interno della coda di operazioni in \ttt{Traveler\_Pool}, e che un thread del pool la esegua (rimuovendola dalla coda).
	
	Tale operazione prevede le seguenti azioni:
		\begin{itemize}
			\item Estrazione \ttt{next\_station} dalla tappa corrente $t$.
			\item Tramite l'interfaccia esposta dalla Stazione \ttt{next\_station} (procedura identificata con ), viene effettuata una richiesta di attsa presso la Piattaforma \ttt{destination\_platform}.
		\end{itemize} 
		
	Internamente alla Stazione, la richiesta si traduce nell'inserimento del Descrittore del Viaggiatore corrente all'interno della coda \ttt{Arrival\_Queue} interna alla Piattaforma \ttt{destination\_platform} selezionata.
	


\newpage
	
