\subsection{Percorrenza di un Viaggiatore}\label{subsec:percorrenza_viaggiatore}
	
	Alla luce della definizione dell'entità Viaggiatore in sezione \ref{subsec:traveler_def}, è possibile definire le azioni che compongono il viaggio:
		\begin{itemize}
			\item Acquisto di un Biglietto (\ttt{Ticket}) presso la Biglietteria della Stazione (\ttt{Ticket\_Office}) di partenza. Ogni Biglietto è composto da una sequenza ordinata di Tappe (\ttt{Ticket\_Stages}), ciascuna contenente:
				\begin{verbatim}
					- start_station
					- next_station
					- train_id 
					- start_platform 
					- destination_platform
					- next_region
				\end{verbatim}
			e da un indice della prossima tappa del Percorso (\ttt{Next\_Stage}).
			
			\item Una volta ottenuto un Ticket, vengono eseguite le seguenti operazioni per ciascuna Tappa del Biglietto:
				\begin{itemize}
					\item Accodamento presso la Piattaforma \ttt{start\_platform} della Stazione \ttt{start\_station} in attesa del Treno \ttt{train\_id}.
					\item All'arrivo del treno \ttt{train\_id}, Accodamento presso la Piattaforma \ttt{destination\_platform} della Stazione \ttt{next\_station} in attesa dell'arrivo di \ttt{train\_id}. 
				\end{itemize}
		\end{itemize} 
	Le azioni sopraelencate possono essere incapsulate in strutture dati che rappresentano una specifica operazione. In questo modo viene generata una \ii{gerarchia di operazioni}, tutte derivanti da un'unica \ii{operazione generica}. Di seguito identificheremo tali operazioni rispettivamente con \ttt{BUY\_TICKET}, \ttt{LEAVE} e \ttt{ARRIVE}.
	
	Il protocollo di operazioni che vengono eseguite da un Viaggiatore, è stato mantenuto il più semplice possibile, e l'intero percorso viene regolato dagli eventi generati dalle entità Treno alla partenza e all'arrivo dalle/nelle Stazioni. 
	
	\subsubsection{Acquisto di un Biglietto}
	
	TODO
	...
	\\
	\\
	
	L'ultima azione eseguita dall'Operazione di acquisto di un biglietto consiste nel caricare nella coda della struttura dati \ttt{Traveler\_Pool} descritta nella sezione \ref{subsec:traveler_def}, l'operazione \ttt{LEAVE} del Viaggiatore corrente.
	
	\subsubsection{Partenza da una Stazione}
	
	Denominiamo $t$ la tappa di indice \ttt{Next\_Stage}, ovvero la tappa corrente. Perché l'operazione di Partenza dalla stazione corrente venga effettuata, è necessaria la precondizione per la quale:
	\begin{itemize}
		\item l'operazione \ttt{LEAVE} è stata inserita nella apposita coda di \ttt{Traveler\_Pool};
		\item in qualche momento essa venga prelevata (rimossa) da tale luogo ed eseguita da uno dei thread nel Pool della struttura. 
	\end{itemize}
Le azioni compiute dall'operazione \ttt{LEAVE} sono:
	
	\begin{itemize}
		\item Estrazione \ttt{start\_station} da $t$;
		\item Tramite l'interfaccia esposta dalla Stazione \ttt{start\_station} (una procedura che identifichiamo con \ttt{Wait\_For\_Train\_To\_Go}), viene richiesto di attendere presso la Piattaforma \ttt{start\_platform}.
	\end{itemize} 
	
	Internamente alla Stazione, tale richiesta viene tradotta nell'inserimento del Viaggiatore, o meglio del suo Descrittore \ttt{Traveler\_Descriptor}, nella coda \ttt{Leaving\_Queue} della Piattaforma indicata da \ttt{start\_platform}.
	
	Nel caso in cui la stazione dalla quale il Viaggiatore vuole partire sia una Stazione di Gateway, allora la procedura di accodamento presso la Piattafroma designata diviene:
	\begin{itemize}
		\item Se la Regione di Destinazione \ttt{next\_region} è diversa dalla Regione corrente, allora
			\begin{itemize}
				\item I dati relativi al Viaggiatore (Descrittore e Biglietto) vengono serializzati (\ii{marshalling}).
				\item Viene inviato un messaggio alla prossima Regione (ovvero \ttt{next\_region}, il cui indirizzo viene recuperato dal Server dei Nomi) contente i dati serializzati, destinato alla Stazione di Gateway corrispondente in tale Regione (nodo).
				\item A destinazione, vengono aggiornati i dati relativi al Viaggiatore locale alla Regione e eseguito un accodamento presso la coda \ttt{Leaving\_Queue} della Piattaforma indicata da \ttt{start\_platform}.
			\end{itemize}
	\end{itemize}
	
	Il comportamento descritto non è proprio della operazione \ttt{LEAVE}, ma è contenuto nell'operazione della Piattaforma per l'accodamento del Viaggiatore presso la coda \ttt{Leaving\_Queue}. 
	
	
	
	
	\subsubsection{Arrivo alla Destinazione successiva}
		
	Similmente a quanto descritto per la partenza, l'arrivo a destinazione prevede che l'operazione \ttt{ARRIVE} sia già stata inserita all'interno della coda di operazioni in \ttt{Traveler\_Pool}, e che un thread del pool la esegua (rimuovendola dalla coda). A differenza della partenza, questa operazione può comportare l'accodamento del Viaggiatore presso una Stazione che non appartiene alla Regione (al nodo) corrente, anche se non di Gateway. 
	
	Vengono eseguite le seguenti azioni:
		\begin{itemize}
			\item Estrazione della prossima stazione (\ttt{next\_station}) e della Regione di destinazione (\ttt{next\_region}) dalla tappa corrente $t$. Si presentano ora due casi:
				\begin{itemize}
					\item Se la prossima Regione è la stessa della Regione corrente, allora mediante l'interfaccia esposta dalla Stazione locale \ttt{next\_station}, viene effettuata una richiesta di attesa presso la Piattaforma \\\ttt{destination\_platform}
					\item Se la destinazione è una Stazione appartenente ad un'altra Regione, allora:
						\begin{itemize}
							\item Viene individuato l'indirizzo della Regione remota (richiesto al \ttt{Name\_Server} che gestisce le Regioni).
							\item Vengono serializzati (\ii{marshalling}) i dati relativi al Viaggiatore (Descrittore del Viaggiatore) e al suo Biglietto.
							\item Viene inviato un Messaggio Remoto alla Regione specificata che provvederà ad effettuare l'aggiornamento del corrispondente Viaggiatore, e al suo accodamento presso la Stazione corretta.
						\end{itemize}
				\end{itemize}
		\end{itemize} 
		
	Internamente alla Stazione, la richiesta si traduce nell'inserimento del Descrittore del Viaggiatore corrente all'interno della coda \ttt{Arrival\_Queue} interna alla Piattaforma \ttt{destination\_platform} selezionata.
	
	Si noti che l'eventuale azione di trasferimento del Viaggiatore al nuovo nodo viene effettuata dall'operazione \ttt{ARRIVE} e non dalla stazione all'arrivo del Treno (descritto nella sottosezione \ref{subsubsec:regional_station_access}). Questa scelta è stata fatta per limitare il più possibile l'occupazione da parte di un Treno della Piattaforma (risorsa protetta), ed evitare quindi l'esecuzione di operazioni potenzialmente lunghe come l'invio di un messaggio remoto.

\newpage
	
