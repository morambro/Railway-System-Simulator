\subsection{Segment}
	
	Il package \ttt{Segment} contiene la definizione del tipo \ii{tagged} \ttt{Segment\_Type} che rappresenta il segmento di congiunzione tra due Stazioni introdotto in sezione \ref{subsec:segment_def}. Esso contiene due strutture dati:
	\begin{itemize}
		\item \ttt{Access\_Controller}, oggetto di tipo \ttt{Priority\_Access\_Controller} introdotto nella sezione \ref{subsec:priority_access_controller};
		\item \ttt{Segment\_Monitor}, oggetto di tipo \ttt{Segment\_Access\_Controller}.
	\end{itemize}
	
	\ttt{Segment\_Access\_Controller} è un tipo \ii{protetto}, che serve a realizzare il protocollo di accesso multiplo al Segmento. Esso contiene i seguenti campi dato:
	\begin{itemize}
		
		\item \ttt{Id}: Identificativo univoco del Segmento;
		\item \ttt{Segment\_Max\_Speed}: Velocità massima di percorrenza del Segmento;
		\item \ttt{Current\_Max\_Speed}: Velocità massima alla quale i Treni percorrono il Segmento;
		\item \ttt{Free}: Indica lo stato di occupazione del Segmento, se \ttt{True} il Segmento è da considerarsi occupato, altrimenti libero.
		\item \ttt{Segment\_Length}: Lunghezza del Segmento;
		
		\item \ttt{Current\_Direction}: Direzione di percorrenza corrente;
		
		\item \ttt{Running\_Trains}: Coda di tipo \ttt{Unlimited\_Simple\_Queue} che contiene gli identificativi dei Treni in transito;
		\item \ttt{Trains\_Number}: Mantiene il numero di Treni attualmente in transito;
		
		
		\item \ttt{First\_End,Second\_End}: Stazioni che sono collegate dal Segmento; 
		
		\item \ttt{Can\_Enter\_First\_End, Can\_Enter\_Second\_End}: Variabili booleane utilizzati come guardie per le enties \ttt{Retry\_First\_End} e \ttt{Retry\_Second\_End}.
		
		\item \ttt{Enter\_Retry\_Num}: Numero di Task in attesa per poter accedere alla risorsa protetta.
		
		\item \ttt{Exit\_Retry\_Num}:  Numero di Task in attesa per poter liberare a risorsa protetta. 
		
		\item \ttt{Train\_Entered\_Per\_Direction}: Numero di Treni transitati per la direzione corrente.
		
		
		\item \ttt{Can\_Retry\_Leave}: Variabile booleana utilizata come guardia per regolare l'\ii{entry} \ttt{Retry\_Leave}.

		
		\item \ttt{Trains\_Order}: Coda di tipo \ttt{Unlimited\_Simple\_Queue} che mantiene l'ordine di accesso dei Treni.

		\item \ttt{Retry\_Num}: Variabile intera utilizzata per memorizzare il numero di Task in attesa su di effettuare l'accesso.

		\item \ttt{Can\_Retry}: Variabile booleana utilizzata come guardia per l'\ii{entry} \ttt{Retry\_Enter}.
		
	\end{itemize}
	
	Il tipo \ttt{Segment\_Type} fornisce un'interfaccia pubblica accessibile ai Task \ttt{Train\_Executor\_Task} per regolare \ii{ingresso} e \ii{uscita} nel/dal Segmento rappresentato, come descritto nella sezione \label{subsubsec:segment_access}. Di seguito viene riportata una descrizione dettagliata della sua realizzazione.
	
	\subsubsection{Ingresso nel Segmento}
	
	L'ingresso è realizzato mediate l'utilizzo di un metodo pubblico \ttt{Enter}, la cui definizione è riportata nel listato \ref{code:segment_type_enter}.
\begin{lstlisting}[caption=\small{Metodo \ttt{Enter} del tipo \ttt{Segment\_Type}},label=code:segment_type_enter]
	procedure Enter(
		This        : access Segment_Type;
		To_Add      : in     Positive;
		Max_Speed   :    out Positive;
		Leg_Length  :	 out Positive) is
	begin
		-- Per prima cosa utilizza l'oggetto
		-- Access_Controller per fornire
		-- priorita' di accesso ai Treni
		-- di tipo FB.
		This.Access_Controller.Gain_Access(To_Add);
		-- Una volta ottenuto l'accesso, viene
		-- aggiunto l'identificativo del Treno
		-- alla coda interna a Segment_Monitor,
		-- per definire l'ordine di esecuzione.
		This.Segment_Monitor.Add_Train(To_Add);
		-- Libera la risorsa Access_Controller
		This.Access_Controller.Access_Gained;
		-- Richiesta di accesso al Segmento, 
		-- con Enter.
		This.Segment_Monitor.Enter(
			To_Add,
			Max_Speed,
			Leg_Length);
	end Enter;
\end{lstlisting}
	
	Essa invoca per prima cosa l'\ii{entry} della risorsa protetta \ttt{Access\_Controller}, in modo tale da ottenere accesso preferenziale, in base alla tipologia ti Treno. Una volta superata la barriere rappresentata dalla \ii{entry} \ttt{Gain\_Access}, viene aggiunto l'identificativo del Treno corrente nella coda interna alla risorsa protetta \ttt{Segment\_Monitor}, e quindi liberata la risorsa \ttt{Access\_Controller} con \ttt{Access\_Gained}. A questo punto, il Treno avrà un ordine assegnato e potrà quindi richiedere l'accesso vero e proprio, utilizzando l'\ii{entry} pubblica \ttt{Enter} fornita da \ttt{Segment\_Monitor}, la quale, con l'utilizzo dell'\ii{entry} privata \ttt{Retry}, consente l'esecuzione ordinata dei Task, secondo l'ordine sancito da \ttt{Trains\_Order} (listato \ref{code:impl_segment_monitor_enter}).

\begin{lstlisting}[caption=\small{Meccanismo che garantisce accesso ordinato secondo l'ordine sancito da \ttt{Trains\_Order}.},label=code:impl_segment_monitor_enter]
potected body Segment_Access_Controller is
	...
	entry Retry(
		To_Add      : in     Positive;
		Max_Speed   :    out Positive;
		Leg_Length  :    out Positive) when Can_Retry is
	begin
		-- Decremento del numero di Task che ritentano.
		Retry_Num := Retry_Num - 1;
		-- Se tale numero e' diventato 0, allora la 
		-- guardia viene richiusa.
		if Retry_Num = 0 then
			Can_Retry := False;
		end if;
		-- Infine riaccoda presso Enter per 
		-- ritentare.
		requeue Enter;
	end Retry;
	
	entry Enter(
		To_Add      : in     Positive;
		Max_Speed   : 	 out Positive;
		Leg_Length  :    out Positive) when True is
	begin
		-- Controllo sulla direzione di accesso del Treno.
		if  (Trains.Trains(To_Add).Current_Station /= First_End) 
		and (Trains.Trains(To_Add).Current_Station /= Second_End) 
		then
			raise Bad_Segment_Access_Request_Exception 
				with "...";
		end if;
		-- Se e solo se il Treno corrente e' il primo della coda 
		-- puo' continuare, altrimenti viene riaccodato su Retry.
		if Trains_Order.Get(1) /= To_Add then
			requeue Retry;
		end if;
		declare
			T : Positive;
		begin
			-- Rimuove il primo elemento della coda
			Trains_Order.Dequeue(T);
			-- Se vi e' almeno un Task in attesa su Retry,
			-- viene aperta la guardia.
			if Retry'Count > 0 then
				Retry_Num := Retry'Count;
				Can_Retry := True;
			end if;
		end;
		-- Riaccodamento all'entry privata che realizza la
		-- politica di accesso multiplo.
		requeue Perform_Enter;
	end Enter;
	...
end Segment_Access_Controller;
\end{lstlisting}

Una volta che il controllo di accesso ordinato è stato superato, si può procedere al controllo di ingresso vero e proprio, mediante l'\ii{entry} privata \ttt{Perform\_Enter}. Essa controlla in primo luogo se il Treno accede dall'estremo \ttt{First\_End} o \ttt{Second\_End}, per poi poter proseguire all'accesso di conseguenza. Nel listato \ref{code:segment_monitor_perform_enter} riportato solo il codice per uno dei due casi, ovvero il caso in cui \ttt{Trains.Trains(To\_Add).Current\_Station = First\_End}, in quanto sono analoghi. 

\begin{lstlisting}[caption=\small{Porzione della \ii{entry} \ttt{Perform\_Enter} per l'accesso dall'estremo \ttt{First\_End}},label=code:segment_monitor_perform_enter]
potected body Segment_Access_Controller is
	...
	entry Perform_Enter(
		To_Add      : in     Positive;
		Max_Speed   :    out Positive;
		Leg_Length  :    out Positive) when True is
	begin
		if Trains.Trains(To_Add).Current_Station = First_End 
		then
			if Free then
				-- Il numero di Treni per direzione viene 
				-- impostato a 1
				Train_Entered_Per_Direction := 1;
				-- Viene impostato ad occupato
				Free := False;
				-- Viene aggiornata la direzione di marcia
				-- corrente, con la Stazione di provenienza 
				-- del Treno.
				Current_Direction := 
					Trains.Trains(To_Add).Current_Station;
			else
				if Trains.Trains(To_Add).Current_Station = 
				   Current_Direction 
				then
					if Train_Entered_Per_Direction = Max then
						-- Se il numero di accessi e' il massimo 
						-- consentito per direzione...
						if Retry_Second_End'Count > 0 then
							-- se vi sono task in attesa dall'estremo
							-- opposto del Segmento, allora il
							-- task corrente viene accodato presso la 
							-- entry Retry_First_End, in attesa che 
							-- arrivi il proprio turno.
							Can_Enter_First_End := False;
							requeue Retry_First_End;
						end if;
					else
						-- se il massimo numero di accessi per 
						-- direzione NON e' stato raggiunto, allora
						-- viene incrementato il numero di accessi
						Train_Entered_Per_Direction := 
							Train_Entered_Per_Direction + 1;
					end if;
				else
					-- nel caso in cui il Treno corrente volesse 
					-- accedere nel senso opposto al senso di 
					-- marcia, dovra' attendere.
					requeue Retry_First_End;
				end if;
		 	end if;
		else
			-- Secondo caso simmetrico.
			...
		end if;	
		-- A questo punto il Treno ha avuto il permesso
		-- di accedere, e vengono effettuate le opportune
		-- inizializzazioni.
		...
			
	end Perform_Enter;
	...
end Segment_Access_Controller;
\end{lstlisting}

Nel caso in cui il Segmento sia libero (\ttt{Free=True}) allora il Treno può accedere al Segmento.
Nel caso in cui invece il Segmento non sia libero, allora viene verificata la possibilità di accesso multiplo, ovvero se:
	\begin{itemize}
		\item La direzione di percorrenza del Treno è la stessa di quella corrente e il numero massimo di accessi per direzione (\ttt{Max}) non è stato raggiunto 
		\item oppure se nessun Task è accodato all'estremo opposto, ovvero sulla \ii{entry} \ttt{Retry\_Second\_End}. 
	\end{itemize}
In tutti gli altri casi il Task corrente viene riaccodato sulla \ii{entry} \ttt{Retry\_First\_End} che avrà guardia booleana chiusa.

Se il Task corrente non è stato riaccodato ad un'altra \ii{entry}, allora viene incrementato di uno il contatore dei Treni in transito, e viene inserito nella coda \ttt{Running\_Trains} l'identificativo del Treno corrente. Vengono infine aggiornati i dati relativi alla velocità di percorrenza (il codice viene omesso per brevità).

	Le \ii{entries} private \ttt{Retry\_First\_End} e \ttt{Retry\_Second\_End} sono molto simili e sono utilizzate per mantenere accodati i Task relativi ai Treni in attesa di accedere al Segmento, rispettivamente presso l'estremo \ttt{First\_End} e \ttt{Second\_End}. Viene riportato nel listato \ref{code:segment_monitor_retry_first} il codice che realizza l'\ii{entry} \ttt{Retry\_First\_End}.
	
\begin{lstlisting}[caption=\small{\ii{entry} utilizzata per accodare i Task che rappresentano Treni in attesa di accesso presso il primo estremo del Segmento.},label=code:segment_monitor_retry_first]
potected body Segment_Access_Controller is
	...
	entry Retry_First_End(
		To_Add     :in		Positive;
		Max_Speed  : 	out Positive;
		Leg_Length :	out	Positive) when Can_Enter_First_End
	is
	begin
		-- Decremento del numero di Task che 
		-- possono ri-tentare l'accesso
		Enter_Retry_Num := Enter_Retry_Num - 1;
		-- una volta che tale numero e' 0, 
		-- la guardia viene richiusa.
		if Enter_Retry_Num = 0 then
			Can_Enter_First_End := False;
		end if;
		-- il nuovo tentativo viene effettuato ri-accodando
		-- il task corrente presso l'entry Enter.
		requeue Enter;
	end Retry_First_End;
	...
end Segment_Access_Controller;
\end{lstlisting}
	
	L'attesa dei task presso questa \ii{entry} è regolata dalla guardia booleana \ttt{Can\_Enter\_First\_End}, mentre il numero di tentativi di nuovo accesso al Segmento, viene regolato dal parametro \ttt{Enter\_Retry\_Num}.
	
	\subsubsection{Uscita dal Segmento}
	
	L'uscita da un Segmento da parte di un Treno, ha il prerequisito fondamentale per cui esso ha prima avuto accesso a tale Segmento, e quindi il suo identificativo univoco sarà contenuto nella coda \ttt{Running\_Trains} della risorsa protetta \ttt{Segment\_Monitor}.
	Il tipo \ii{tagged} \ttt{Segment\_Type} espone il metodo \ttt{Leave} che permette ad un Treno di richiedere l'uscita dal Segmento, il quale semplicemente invoca l'\ii{entry} omonima di \ttt{Segment\_Monitor}.
	Il processo di uscita avviene secondo l'ordine di percorrenza dei Treni all'interno del Segmento.	
	La \ii{entry} \ttt{Leave} della risorsa \ttt{Segment\_Monitor} (listato \ref{code:segment_monitor_leave}), per prima cosa controlla che il Task correntemente in esecuzione sia effettivamente il prossimo a dover uscire dal Segmento. Tale controllo viene effettuato confrontando il primo elemento della coda \ttt{Running\_Trains} con l'identificativo del Treno rappresentato dal Task in esecuzione: se essi sono uguali allora l'identificativo viene rimosso dalla coda e viene effettuata l'uscita, altrimenti il Task corrente viene accodato presso l'\ii{entry} privata \ttt{Retry\_Leave}.
	
\begin{lstlisting}[caption=\small{Uscita di un Task dal Segmento.},label=code:segment_monitor_leave]
potected body Segment_Access_Controller is
	...
	
	entry Retry_Leave(
		Train_D : in Positive) when Can_Retry_Leave is
	begin
		-- Decremento del numero di task che
		-- possono ritentare l'uscita.
		Retry_Num := Retry_Num - 1;
		-- una volta che tale numero e' 0, 
		-- la guardia viene richiusa per permettere
		-- nuovo accodamento
		if(Retry_Num = 0) then
			Can_Retry_Leave := False;
		end if;
		-- infine viene riaccodato il task corrente
		-- presso Leave, per ritentare l'uscita.
		requeue Leave;
	end Retry_Leave;
	
	entry Leave(
		Train_D : in Positive) when not Free is
	begin
		if Running_Trains.Get(1) = Trains.Trains(Train_D).ID 
		then
			-- il primo elemento della coda viene rimosso
			declare
				T : Positive;
			begin
				Running_Trains.Dequeue(T);
			end;
			-- Se c'e' almeno un Task in attesa presso la
			-- guardia di uscita, essa puo' essere aperta
			-- per permettere al prossimo Task di uscire.
			if(Retry_Leave'Count > 0) then
				-- viene memorizzato il numero di task
				-- che possono tentare l'uscita.
				Exit_Retry_Num := Retry_Leave'Count;
				Can_Retry_Leave := True;
			end if;
			-- Decremento del numero di Treni che 
			-- percorrono il Segmento.
			Trains_Number := Trains_Number - 1;
			if Trains_Number = 0 then
				-- Nel caso in cui il Semento si sia
				-- svuotato, la velocita' massima
				-- di percorrenza viene aggiornata 
				Current_Max_Speed := Segment_Max_Speed;
				-- Inoltre se vi sono treni in attesa di 
				-- accedere dalla direzione opposta, 
				-- la guardia viene aperta.
				if Current_Direction = First_End then
					-- caso in cui la direzione corrente sia 
					-- proveniente dal primo estremo del segmento.
					if Retry_Second_End'Count > 0 then
						-- se ci sono task in attesa presso la
						-- entry Retry_Second_End, allora essi 
						-- potranno riprovare l'accesso.
						Enter_Retry_Num := Retry_Second_End'Count;
						-- viene aperta la guardia booleana
						Can_Enter_Second_End := True;
						Can_Enter_First_End := False;
					else
						-- se non vi sono treni in attesa, allore
						-- il Segmento viene dichiarato libero.
						Free := True;
					end if;
				else
					-- Caso simmetrico
					...
				end if;
				-- Viene posto a libero il segmento.
				Free := True;
			end if;	
			-- Viene aggiunto il Treno alla coda di arrivo 
			-- presso la prossima Stazione.
			if Current_Direction /= First_End then
				Environment.Stations(First_End).Add_Train(
					Train_ID 	=> Train_D,
					Segment_ID	=> Id);
			else
				Environment.Stations(Second_End).Add_Train(
					Train_ID 	=> Train_D,
					Segment_ID	=> Id);
			end if;
		else
			-- Requeue alla entry Retry_Leave per
			-- rispettare l'ordine di percorrenza.
			requeue Retry_Leave;
		end if;
	end Leave;
	...
end Segment_Access_Controller;
\end{lstlisting}
	 
	L'uscita viene completata nel seguente modo: per prima cosa viene verificata l'eventuale presenza di Task in attesa sulla guardia (chiusa) della \ii{entry} \ttt{Retry\_Leave}; successivamente, viene verificato se il Segmento è vuoto o se vi sono altri Treni in percorrenza. Nel primo caso, viene verificata l'eventuale presenza di Treni (Task) in attesa presso l'estremità opposta del Segmento, e in tal caso la loro guardia viene aperta per permettere di ritentare l'accesso. Se il Segmento è vuoto viene re-impostato il valore di \ttt{Free} a \ttt{True}.	
	Infine, deve essere comunicato l'ordine di uscita alla successiva Stazione. Questa operazione si traduce nell'invocazione della procedura \ttt{Add\_Train} messa a disposizione dall'interfaccia \ttt{Station\_Interface} del package \ttt{Generic\_Station}. 

La soluzione presentata sfrutta gli strumenti offerti dal linguaggio Ada per garantire una semantica adeguata di accesso, percorrenza e uscita.
