\documentclass[12pt,a4paper]{report}
\usepackage{amsmath}
\usepackage{amsfonts}
\usepackage{amssymb}
\usepackage{array}
\usepackage{fancyhdr}
\usepackage{lastpage} 
\usepackage{longtable}
\usepackage[pdftex]{graphicx}
\usepackage[colorlinks=true,citecolor=black,urlcolor=black,linkcolor=black,breaklinks]{hyperref}
\usepackage{glossaries}
\usepackage{ucs}
\usepackage[utf8x]{inputenc}
\usepackage[italian]{babel}
\usepackage{multirow}
\usepackage{amsmath,amssymb,amsfonts,textcomp}
\usepackage{color}
\usepackage{supertabular}
\usepackage{hhline}
\usepackage{listings}
\usepackage{eso-pic,graphicx}
\usepackage{verbatim}
\usepackage{changepage}
\usepackage{ifthen}

\title{\Huge Railway Simulation}
\author	{
	\Large Moreno Ambrosin\\
	\Large mat. 1035635\\
	[1.5cm]
	\Large Progetto Sistemi Concorrenti e Distribuiti\\
	[3cm]
	\large Corso di laurea magistrale in Informatica \\
	\large Università degli Studi di Padova\\
	\large Padova\\
	[1.5cm]
		}
\date{\today}

\fancypagestyle{content}{
	\setcounter{page}{1}
	% \leftmark Per avere il titolo del capitolo, usare questo al posto di rightmark 
	\lhead{\footnotesize{\rightmark}}
	\chead{}
	\rhead{}
	%\rhead{\includegraphics[scale=0.5]{imgs/logoUnipdAltoDes.png}}
	\lfoot{\varTitle\ \varVersion}
	\lfoot{\footnotesize{Relazione progetto SCD}}
	\cfoot{}
	\rfoot{Pagina \thepage\ di \pageref{mylastpage}}

	\pagenumbering{arabic}
}

\newcommand{\ii}[1]{\textit{#1}}
\newcommand{\bb}[1]{\textbf{#1}}
\newcommand{\ttt}[1]{\texttt{#1}}

% ############################################### DOCUMENT ###########################################

\begin{document}
	\maketitle
	\tableofcontents
	\pagestyle{content}
	\chapter{Il problema}

Il progetto didattico prevede la progettazione e la realizzazione di un simulatore software di un sistema ferroviario. Tale sistema prevede i seguenti requisiti di base (la specifica originale è consultabile all'indirizzo \url{http://www.math.unipd.it/~tullio/SCD/2005/Progetto.html}):
	\begin{itemize}
		\item La presenza di Treni appartenenti a categorie a priorità e modalità di fruizione diverse. 
		\item La presenza di Viaggiatori, che eseguono operazioni elementari come acquisto di un biglietto, salita/discesa su/da un Treno, ecc.
		\item La definizione di un ambiente costituito da un insieme di Stazioni, ciascuna composta da:
			\begin{itemize}
				\item Piattaforme di attesa per Treni e Viaggiatori.
				\item Un Pannello Informativo che visualizza informazioni sui Treni in arrivo, in transito e che hanno appena superato la Stazione corrente.
				\item Una Biglietteria, presso la quale un Viaggiatore può acquistare un biglietto di viaggio.
			\end{itemize}
		\item La presenza di Segmenti di collegamento tra Stazioni, a percorrenza bidirezionale.
		\item La presenza di un entità di controllo globale che mantiene lo stato di ciascun Viaggiatore e ciascun Treno in transito.
		\item L'obbligo da parte di un Viaggiatore di acquistare un Biglietto prima di poter usufruire del servizio ferroviario.
		\item La presenza di collegamenti multipli tra Stazioni, ovvero ciascuna Stazione può essere raggiunta da più Segmenti e da ciascuna stazione possono partire più segmenti.
		\item Il percorso portato a termine da un Viaggiatore può comprendere cambi di treno.
		\item Ciascun Treno appartiene ad una delle seguenti due categorie:
			\begin{description}
				\item {\bb{Regionale}}\\Treno a bassa priorità, senza posto garantito.
				\item {\bb{FB}}\\Treno a priorità più alta, che necessita prenotazione.
			\end{description}
		\item Ciascun Treno possiede una capienza massima di Viaggiatori.
	\end{itemize}

	\chapter{Analisi Concorrenza e Distribuzione}

La modellazione di un sistema ferroviario presenta diverse problematiche relative alla concorrenza e alla distribuzione. Di seguito verranno presentate le più rilevanti.

\section{Ingresso in un Segmento da parte di un Treno}\label{ingresso_segmento}

\begin{figure}[htbp]
	\includegraphics[width=\textwidth,keepaspectratio]{imgs/ingresso_segmento.pdf}
	\caption{\footnotesize{Accesso ad un Segmento da Parte di uno o più Treni.}}
\end{figure}

L'accesso ad un segmento di collegamento tra due stazioni da parte di un Treno, è inerentemente concorrente. Tale azione presenta infatti i seguenti requisiti:
	\begin{itemize}
		\item L'ingresso presso un Segmento deve avvenire in \ii{mutua esclusione}; è infatti impossibile che due o più entità Treno accedano ad uno stesso Segmento contemporaneamente.
		\item Più Treni possono circolare su un Segmento contemporaneamente. Questo comporta:
			\begin{itemize}
				\item il mantenimento di un ordine di ingresso al Segmento;
				\item la regolazione della velocità di transito di ciascun Treno in base alla velocità di quelli che lo precedono;
				\item l'impossibilità di un Treno di accedere ad un Segmento qualora vi siano altri Treni che lo percorrono in senso opposto.
			\end{itemize}
		\item Dev'essere data precedenza d'accesso al Segmento, ai Treni di tipo \ttt{FB}.
	\end{itemize}


\section{Uscita da un Segmento e accesso alla Stazione}

\begin{figure}[htbp]
	\includegraphics[width=\textwidth,keepaspectratio]{imgs/Ingresso_Stazione.pdf}
	\caption{\footnotesize{Uscita da un Segmento e accesso ad una Piattaforma di uno o più Treni.}}
\end{figure}

Il problema introdotto in sezione \ref{ingresso_segmento}, vincola i requisiti che dovranno essere soddisfatti relativamente all'uscita da un Segmento e conseguente accesso alla Stazione successiva. Per queanto riguarda l'uscita dal Segmento avremo quindi i seguenti requisiti:
	\begin{itemize}
		\item L'ordine di ingresso al Segmento dev'essere mantenuto all'uscita. 
		\item L'ordine di uscita da un Segmento dev'essere mantenuto nell'accesso alla stazione successiva da parte dei Treni in transito.
	\end{itemize}

L'ingresso in una Stazione, permette ad un Treno di occupare una della Piattaforme disponibili e, dato che da specifica una stazione può essere raggiunta da più Segmenti, tale azione sarà svolta in modo concorrente tra i treni in uscita dai vari Segmenti, secondo i seguenti vincoli:
	\begin{itemize}
		\item I Treni di tipo \ttt{FB} avranno priorità maggiore nell'occupare una Piattaforma.
		\item Ciascuna Piattaforma è acceduta in mutua esclusione, e può essere occupata da un solo Treno alla volta.  
	\end{itemize}

Questo significa che l'accesso ad una Piattaforma sarà ordinato, per tutti i Treni provenienti dallo stesso Segmento, in base all'ordine di uscita da quest'ultimo, e concorrente tra Treni provenienti da Segmenti diversi. 

\section{Acquisto di un Biglietto da parte di un Viaggiatore}

Ciascun Viaggiatore deve acquistare un biglietto prima di poter usufruire dei servizi ferroviari. Per fare ciò, all'interno ogni stazione vi è una Biglietteria presso la quale l'acquisto può essere effettuato. Un biglietto sarà composto da una serie di Tappe, ciascuna relativa ad un tratto del percorso da portare a termine con uno specifico Treno, sia Regionale che FB. 
L'assegnazione di un Biglietto ad un Viaggiatore è semplice se il suo percorso prevede solo l'utilizzo di Treni Regionli, mentre è più complesso se vi sono tappe da raggiungere con Treni FB, i cui biglietti vengono erogari se e soltanto se vi è ancora posto all'interno del treno.
\'E quindi necessario prevedere l'esistenza di una \ii{biglietteria centrale} che mantenga le prenotazioni dei Treni di tipo FB. Si ricavano quindi i seguenti requisiti:
	\begin{itemize}
		\item La biglietteria centrale va acceduta in mutua esclusione, in modo da evitare prenotazioni inconsistenti di Biglietti.
		\item L'erogazione di biglietti può fallire in caso non vi siano posti per percorrere alcune tappe; il sistema deve reagire di conseguenza.
	\end{itemize}


	\chapter{Soluzione}

Di seguito, verrà presentata la soluzione realizzata, in termini di progettazione come sistema distribuito e concorrente.

\section{Architettura Logica di Distribuzione}

Un diagramma infromale delle componenti distribuite che compongono il sistema è presentato in figura X. Di seguito verranno descritte le principali.
	
	\subsection{Regioni}\label{sec:distr_regioni}
	
	La simulazione è stata suddivisa in Regioni, le quali risiederanno su nodi di calcolo diversi. Questa scelta aggiunge i seguenti requisiti minimi alla specifica iniziale:
	\begin{itemize}
		\item I Treni, se previsto dal percorso, possono viaggiare da una Regione all'altra.
		\item I Passeggeri possono raggiungere destinazioni in Regioni diverse.
		\item Esisteranno stazioni che chiameremo di Gateway che permettono a Treni e Passeggeri di raggiungere Regioni diverse. 
		\item Deve essere garantita consistenza temporale nel passaggio da una Regione ad un'altra.
	\end{itemize}
	Da questa scelta consegue inoltre l'introduzione di un semplice Server dei Nomi che mantiene traccia di ciascuna Regione, in modo tale da rendere agevole la risoluzione della locazione alla quale l'entità si trova. 
	
	
	
	\subsection{Biglietterie}
	
	Per poter gestire meglio la definizione di un percorso e l'erogazione di un Biglietto per un Viaggiatore, ho pensato di introdurre una gerarchia su due livelli, di Biglietterie. Ci saranno dunque tre categorie di biglietterie:
		\begin{description}
			\item {\bb{Biglietterie di Stazione}} \\
			Forniscono un'interfaccia adeguata ai Viaggiatori per poter acquistare un biglietto.
			\item {\bb{Biglietterie Regionali}}\\
			Hanno conoscenza regionale della topologia del grafo composto da Stazioni e Segmenti.
			\item {\bb{Biglietteria Cantrale}} \\ 
			Ha conoscenza di più alto livello; in particolare, essa mantiene traccia delle connessioni tra le varie regioni ( ovvero i collegamenti tra Stazioni di Gateway di regioni diverse).
		\end{description} 
	
	\subsection{Controller Centrale}
		
	Il Conterollo Centrale è una entità distribuita, alla quale tutti i nodi inviano Eventi per notificare lo stato di avanzamento globale della simulazione. Esso fornisce una interfaccia alle varie Regioni per ricevere gli Eventi, ed un'interfaccia per permettere a client remoti di poter visualizzare gli effetti di tali Eventi. Quest'ultima possibilità è ottenuta mediante un meccanismo di tipo Publish/Subscribe, attraverso il quale client remoti possono registrarsi presso il Controller per ricevere, in modalità Push, gli Eventi.
	In questo modo è possibile per un qualsiasi client interfacciarsi al Controller e fornire, ad esempio, una rappresentazione grafica della simulazione.

\newpage
\section{Architettura Logica Concorrente}

Internamente a ciascuna Regione di simulazione, sono definite entità che modellano le interazioni previste dalla specifica del problema. Di seguito sono descritte le principali.

	
	%  ################################################# SEGMENTO ######################################################
	\subsection{Segmento}
	
	Un Segmento (\ttt{Segment}) è modellato come una \ii{entità reattiva con agente di controllo a molteplicità N}, dove \ttt{N} è il numero massimo di utilizzatori, la quale fornisce un'interfaccia utilizzabile da entità attive di tipo Treno per accedere ed uscire in maniera ordinata. Esso è caratterizzato da:
			\begin{itemize}
				\item le due stazioni che esso collega;
				\item una lunghezza;
				\item una velocità massima di percorrenza.
			\end{itemize}


	%  ################################################# VIAGGIATORE ######################################################
%		\item {\bb{Viaggiatore}} \\
	\subsection{Viaggiatore}			
	
	In prima analisi, un Viaggiatore \ttt{Traveler} può essere modellato come una \ii{entità Attiva} che esegue le seguenti semplici operazioni:
		\begin{itemize}
			\item Acquisto di un Biglietto (\ttt{Ticket}) presso la Biglietteria della Stazione (\ttt{Ticket\_Office}) di partenza. Ciascun Biglietto è composto da una sequenza ordinata di Tappe (\ttt{Ticket\_Stages}), ciascuna composta da:
				\begin{verbatim}
					- start_station
					- next_station
					- train_id 
					- start_platform 
					- destination_platform
				\end{verbatim}
			\item Una volta ottenuto un Ticket, vengono eseguite le seguenti operazioni per ciascuna Tappa del Biglietto:
				\begin{itemize}
					\item Accodamento presso la Piattaforma \ttt{start\_platform} della Stazione \ttt{start\_station} in attesa del Treno \ttt{train\_id}.
					\item All'arrivo del treno \ttt{train\_id}, Accodamento presso la Piattaforma \ttt{destination\_platform} della Stazione \ttt{next\_station} in attesa dell'arrivo di \ttt{train\_id}. 
				\end{itemize}
		\end{itemize} 
	
	La modellazione dell'entità Viaggiatore come Attiva non può però essere semplicemente associata ad un processo, soprattutto in presenza di un modello di distribuzione come quello presentato in sezione \ref{sec:distr_regioni}.
	In questa ipotesi infatti, nel caso in cui un Viaggiatore si spostasse da una Regione ad un'altre, avremmo a disposizione solo due possibili soluzioni:
		\begin{itemize}
			\item La migrazione del processo che rappresenta il Viaggiatore sul nodo (Regioni) di destinazione, come distruzione del processo sul nodo di partenza e creazione dinamica dello stesso sul nodo destinazione. Questa operazione è in generale computazionalmente molto costosa. 
			\item La replicazione del processo che rappresenta il Viaggiatore su tutti i nodi, e l'attivazione, intesa come cambio di stato del processo in modo tale che possa competere per la CPU; tale soluzione è molto costosa in termini di memoria utilizzata, e non scala all'aumentare del numero di passeggeri.
		\end{itemize}
	La soluzione adottata consiste nel disaccoppiare le operazioni svolte da ciascun Viaggiatore dal processo che le esegue, prevedendo una struttura dati costituita da:
		\begin{itemize}
			\item un \ii{pool di M processi} dimensionato in maniera opportuna;
			\item una \ii{coda di operazioni} che man mano vengono estratte ed eseguite dai processi nel pool.
		\end{itemize}
	In questo modo è sufficiente replicare per ciascun Viaggiatore, su tutti i nodi che compongono il sistema, una struttura dati che contiene i suoi dati e le operazioni che esso eseguirà. 
	Il cambio di Regione di un Viaggiatore sarà quindi potrà essere ottenuto semplicemente inserendo la prossima operazione da eseguire per il Viaggiatore nella coda del pool di processi del nodo destinazione.  
		
	%  ################################################# TRENO ######################################################
	\subsection{Treno}
			Un Treno (\ttt{Train}) è una \ii{entità Attiva}, la quale esegue ciclicamente un numero finito di operazioni. A ciascuna entità Treno, è assegnato un Percorso (\ttt{Route}) di andata e un Percorso di ritorno, ovvero sequenze di Tappe (\ttt{Stage}) successive, ciascuna composta dalla n-upla:
				\begin{center}
					\begin{verbatim}
						       <next_segment,next_station,next_platfom,action>
					\end{verbatim}
				\end{center}
dove \ttt{action} indica quello che un Treno dovrà compiere presso la prossima stazione, a scelta tra \ttt{ENTER}, per entrare ed effettuare discesa e salita passeggeri o \ttt{PASS} per non fermarsi e oltrepassare la Stazione.

Le operazioni effettuate sono, per ciascuna Tappa del Percorso corrente (di andata o di ritorno):
				\begin{itemize}
					\item Accesso al prossimo Segmento \ttt{next\_segment} previsto.
					\item Percorrenza all'interno del Segmento come attesa finita di durata proporzionale alla lunghezza del Segmento e alla velocità massima alla quale il Treno può percorrerlo.
					\item Uscita dal Segmento e richiesta di Accesso alla Stazione successiva (\ttt{next\_station}) presso la Piattaforma indicata da \ttt{next\_platfom}, per eseguire l'azione \ttt{action}.
					\item Se \ttt{action = ENTER} allore effettua discesa e salita dei Viaggiatori in attesa dell'arrivo del Treno.
					\item Uscita dalla Piattaforma \ttt{next\_platfom}.
				\end{itemize}
				
	%  ################################################# STAZIONE ######################################################
	\subsection{Stazione}
	
	Una Stazione (\ttt{Station}) è modellata come una struttura dati contenete:
		\begin{itemize}
			\item un certo numero n > 2 di Piattaforme (\ttt{Platform});
			\item una Biglietteria (\ttt{Ticket\_Office});
			\item un Pannello Informativo (\ttt{Notice\_Panel}).
		\end{itemize}
	Essa offre una interfaccia alle entità Treno e Viaggiatore per l'accesso a Piattaforme e Biglietteria.
		
		\subsubsection{Piattaforma}
	
		Una Piattaforma è modellata come una \ii{entità reattiva con agente di controllo, a molteplicità 1}. Essa espone un'interfaccia che permette alle entità Treno di potervi sostare ed effettuare discesa e salita delle entità Passeggero o di poter superare la stazione, e alle entità Passeggero di accodarsi in attesa di uno specifico Treno.
		
		\subsubsection{Biglietteria}
		
		Una Piattaforma è modellata come una interfaccia che permette al Viaggiatore di acquisire un Biglietto di viaggio. 
		
		\subsubsection{Pannello Informativo}
		
		Una Piattaforma è modellata come una \ii{entità reattiva con agente di controllo, a molteplicità 1}. Esso espone una interfaccia tale da permettere alla stazione di notificare lo stato delle entità Treno che stanno arrivando, quelle in sosta e quelle in partenza.

\section{Interazione tra le Entità}

	\subsection{Viaggio di un Viaggiatore}
	
	\subsection{Viaggio di un Treno}
	
	
	


	
%3 - Dimostrazione che la soluzione risolve il problema
 
   \label{mylastpage}
\end{document}
